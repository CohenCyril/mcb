\chapter{Writing Statements/specifications}
THIS CHAPTER IS ABOUT METHODOLOGY

% Where one learns to do proofs.
% Boolean reflection in practice, views, discussion on the definition of leq,
% proofs on things defined in the
% previous chapter, associated tactics, exercises on prime, div,
% binomial, etc.
% 
% spec? A new vernacular to declare specs without typing coinductive and
% by writing explicitly the equations.

Discussion prop/bool, intuitionism, extraction (we should be able to avoid talking about impredicativity, but use Prop for computationally irrelevant).

\begin{itemize}
\item can we specify all we have written so far using just = and forall? No.
	Example dvdn needs exists to be specified
\item exists, and, or, neg, False, True as inductives (again CH style)
\item related tactics: split, left, right,exists,case
\end{itemize}

Anyway to take advantage of computation (ssr style) we want to
work with bool as much as possible:

\begin{itemize}
\item reflect is the right way to write iff, <->, specialized to bool
	so that the proof language recognizes it and offer a bit more ergonomic
\item is-true
\item infrastructure for reflect: iffp, altp
\item no split if goal is \&\& (metodology)
\end{itemize}

Writing good statements

\begin{itemize}
\item = as iff for bool, because rewrite is easy to use
\item and3p, spec (drive your proof), 
\item HO "predicators": commutative.  Discuss order of quantification in
	transitive and similar, naming conventions
\item advanced stuff: classically P instead of not-not P
\item in general good quantifications and implicit arguments and good library
	makes it possible to work without evars
\item . \C{<=} . ?= iff .
\end{itemize}

Statements do also occur in the middle of proofs.  There we have many ways to
write them compactly, wlog and have.

Comparison with other possible ways of writing properties:
\begin{itemize}
\item impact of le v.s. leq in a proof
\end{itemize}

In this chapter we should distill a description of our
systematic-reactions, reflexes, to typical situations a
beginner would screw up.

