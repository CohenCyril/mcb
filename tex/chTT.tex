\Chapter{Type Theory}{The Curry-Howard correspondence}
\label{sec:ttch}

By now, the informed reader is likely to wonder when the authors of the book
will eventually mention the mathematical foundation of Coq, a variant of
type theory called \mcbCIC{}, and
in particular explain the Curry-Howard isomorphism.  Indeed this text departs
from the standard presentation of Coq, that typically starts by
presenting the logic and showing how standard connectives like
conjunction or disjunction are defined in term of inductive types.

We made a deliberate choice to put forward the programming aspects of
type theory and how programs can be used both to express computable functions
and decidable predicates or boolean connectives.
Indeed this formalization choice is one of the main ingredients of the \mcbMC{}
library.  The computational behavior of program is at the core of
type theory, i.e. the fact that a function computes to a value requires no
proof.  This fact provided a powerful form of automation, and as we will
see later will also ease the construction of ``sub sets''.

Now we really need to provide a minimal presentation of the statements as types
correspondence.  The universe of boolean predicates is in \mcbMC{} limited: one
cannot for example express a general $\exists$ quantification, and also the
status of the $\forall$ quantification has been so far never really explained.
This presentation is far from being exhaustive, the interested reader can find
a modern presentation of type theory in \cite{hottbook}.

\section{Connectives}

\subsection{Primitive types and terms formers}\label{sec:chi}

In type theory we say that \emph{logical statements} correspond to
\emph{types} and that \emph{proofs} correspond to \emph{programs}.

The simplest example one could think of is the formal statement saying that
a proposition $A$ implies itself, namely $A \rightarrow A$.  If one reads
the implication symbol $\to$ as the type of functions, the statements reads
as a function from $A$ to $A$.  A program with that type would be
\C{(fun x : A => x)}.  Such program takes in input a term $x$ of
type $A$, that we here see as a proof of $A$, and returns it, i.e. it
produces in output a proof of $A$.  We say that such program is a
proof of $A \rightarrow A$.

What is striking here is that the same class of terms represents both programs
and proofs, for example the identity function of natural numbers has the very
same shape of the proof we've just seen.  So we are left with only two
concepts, types and terms, playing a double role.  We now see how types and
terms can be formed.

Regarding types, the primitive type formers we have seen are the function space
$\to$, also logical implication, and the universal quantification
$\forall$, used to describe polymorphic function as well as parametric lemmas
(like induction principles).
Standard logical rules describing how one can prove formulas involving these
connectives are:

\begin{center}
\AxiomC{$A$}
\noLine
\UnaryInfC{$\vdots$}
\noLine
\UnaryInfC{$B$}
\RightLabel{$\to_I$}
\UnaryInfC{$A \to B$}
\DisplayProof
\hspace{1cm}
\AxiomC{$B$}
\RightLabel{$\forall_I$ ($x$ fresh)}
\UnaryInfC{$\forall x, B$}
\DisplayProof
\end{center}

The former reads: to prove $A \to B$ one can prove $B$ under the
assumption $A$.  The latter: to prove $\forall x,B$ one can prove $B$
assuming that $x$ is fresh.  We annotate such rules with the terms
corresponding to these proof rules.

\begin{center}
\AxiomC{\C{x : }$~A$}
\noLine
\UnaryInfC{$\vdots$}
\noLine
\UnaryInfC{\C{b : }$~B$}
\RightLabel{$\to_I$}
\UnaryInfC{\C{(fun x : A => b) : }$~A \to B$}
\DisplayProof
\hspace{1cm}
\AxiomC{\C{b : }$B$}
\RightLabel{$\forall_I$}
\UnaryInfC{\C{(fun x : A => b) : }$~\forall x, B$}
\DisplayProof
\end{center}

The anonymous function constructor \C{(fun .. => ..)} serves as a proof
for both rules: the term \C{b} is in both cases a proof of $B$.  The only
difference is that $x$ is only allowed to occur in $B$ in the second rule.
Said otherwise in type theory $\to$ is a special case of $\forall$ where
the bound variable does not occur in the quantified formula.  In other
words there is no semantic difference between \C{(nat -> bool)} and
\C{(forall x : nat, bool)}.

Logical connectives come with elimination rules, in particular

\begin{center}
\AxiomC{$A \to B$}
\AxiomC{$A$}
\RightLabel{$\to_E$}
\BinaryInfC{$B$}
\DisplayProof
\hspace{1cm}
\AxiomC{$\forall x, B$}
\RightLabel{$\forall_E$ ($t$ a term)}
\UnaryInfC{$B[t/x]$}
\DisplayProof
\end{center}

Function application serves as a proof for both rules.

\begin{center}
\AxiomC{\C{f : }$~A \to B$}
\AxiomC{\C{a : }$~A$}
\RightLabel{$\to_E$}
\BinaryInfC{\C{(f a) :}$~B$}
\DisplayProof
\hspace{1cm}
\AxiomC{\C{f : }$~\forall x : A, B$}
\AxiomC{\C{t : }$~A$}
\RightLabel{$\forall_E$}
\BinaryInfC{\C{(f a) : }$~B[t/x]$}
\DisplayProof
\end{center}

Here the programs as proofs correspondence has a visible impact in the proofs
part of the \mcbMC{} library.  In particular quantified lemmas, being programs,
can be instantiated by simply passing arguments to them.  Exactly as one can
pass \C{3} to \C{addn} and obtain \C{(addn 3)}, the function adding three, one
can ``pass'' \C{3} to the lemma \C{addnC} and obtain a proof of the statement
\C{(forall y, 3 + y = y + 3)}.  Remark that the argument passed to \C{addnC}
shows up in the type of the resulting term \C{(addnC 3)}:  the type of the
\C{addnC} program depends on the value the program is applied to.  That is the
difference between the \emph{dependent function space} ($\forall$)
and the standard function space ($\to$).

Finally note that the $\forall$ quantification specifies the type of the bound
variable, and that the $\forall_E$ checks the term $t$ has the right type.

\subsection{Inductive types}\label{ssec:indtypes}

In the previous chapters we used many data types already.  For example we have
seen the type \C{nat} and its constructors \C{O} and \C{S}.  Through the looking
glasses of the Curry-Howard correspondence \C{O}, being a term of type \C{nat}
can represent a ``proof'' of \C{nat}.  For data types we prefer to say
\emph{inhabited} rather than \emph{proved}, but there is no real difference.

Inductive ``data'' types can be used to represent logical connectives, in
particular the ones we miss so far, lie the existential quantifier.

\begin{center}
\AxiomC{$A$} \AxiomC{$B$}
\RightLabel{$\wedge_I$}
\BinaryInfC{$A \wedge B$}
\DisplayProof
\hspace{1cm}
\AxiomC{$A \wedge B$}
\RightLabel{$\wedge_E$ (left)}
\UnaryInfC{$A$}
\DisplayProof
\end{center}

The first rule reads: to prove $A \wedge B$ one needs to prove both
$A$ and $B$.  The latter lets one prove $A$ whenever one can prove
the stronger $A \wedge B$ statement.

In Coq such connective can be characterized by the following
inductive definition.

\begin{coq}{name=And}{}
Inductive and (A B : Prop) : Prop := conj (pa : A) (pb : B).
Notation "A /\ B" := (and A B).
\end{coq}
\index[vernac]{\C{Inductive} (predicate)}

Remark that the ``data'' type \C{And} is tagged as \C{Prop}, i.e.  we declare
the intention to use it as a logical connective rather than a data type.  The
only constructor \C{conj} that can be used to inhabit \C{and} takes two
arguments, a proof of \C{A} and a proof of \C{B}, faithfully modelling the
logical rule \C{$\wedge_I$}.  Note that \C{A} and \C{B} are parameters of the
inductive definition, similarly to the definition of polymorphic lists
or pairs.  

It is worth to note that the definition of the pair data type,
Exercise~\ref{ex:pair}, is almost identical to the one of \C{and}.

\begin{coq}{}{}
Inductive prod (A B : Type) := pair (a : A) (b : B).
\end{coq}
The striking similarity witnesses how programs (and data) are used in \Coq{}
to represent proofs.

Pattern matching provides a way to express the elimination rule for
conjunction.  In particular the elimination rule is just a projection.

\begin{coq}{name=Ande1}{}
Definition proj1 A B (p : A /\ B) : A :=
  match p with conj a _ => a end.
\end{coq}

Now recall the similarity between $\to$ and $\forall$, where the former is the
simple, non dependent, case of the latter.  If we check the type of
the \C{conj} constructor

\begin{coq}{name=Ande1}{width=4cm}
About conj. 
\end{coq}
\begin{coqout}{}{width=8cm}
aonj: forall A B : Prop, A -> B -> A /\ B
\end{coqout}
we may wonder what happens if the type of the second argument (i.e. \C{B}) is
made dependent on the value of the first argument (of type \C{A}).
What we obtain is the inductive definition corresponding to the
existential quantification.

\begin{coq}{name=Ande1}{}
Inductive ex (A : Type) (P : A -> Prop) : Prop :=
  ex_intro (x : A) (p : P x).
Notation "'exists' x : A , p" := (ex A (fun x : A => p)).
\end{coq}

The \C{ex\_intro} constructor is the only mean to prove a statement like
\C{(exists n, prime n)}.  In such case the first argument would be a number
\C{n} of type \C{nat} while the second argument would be a proof \C{p} of type
\C{(prime n)}.  The \C{ex} inductive definition is again pretty similar to the
pair, but note that the type of the second component depends on the value of
the first one.  Last note the parameter \C{P} that is a function
representing an arbitrary predicate over a term of
type \C{A}.  Hence \C{(P a)} is the instance of the predicate to \C{a}.  E.g.
the predicate of being an even prime number is expressed as
\C{(fun x : nat => ~~ odd x && prime x)}, and the statement expressing the
existence of such number is
\C{(ex nat (fun x : nat => ~~ odd x && prime x))}.

We quickly see the inductive definition of the disjunction.

\begin{coq}{name=Or}{}
Inductive or (A B : Prop) : Prop :=
  | or_introl (a : A)
  | or_intror (b : B).
Notation "A \/ B" := (or A B).
\end{coq}

The elimination rule can again be expressed by pattern matching:

\begin{coq}{name=Or}{}
Definition or_ind (A B P : Prop)
  (aob : A \/ B) (pa : A -> P) (pb : B -> P) : P :=
  match aob with
  | or_introl a => pa a
  | or_intror b => pb b
  end.
\end{coq}

The detail worth noting here is that the pattern match construct has two
branches in this case.  Each branch represents a distinct sub proof.  In this
case to prove \C{P} starting from \C{A \\/ B} one has to deal with all
possibilities: prove \C{P} under the assumption \C{A} and prove \C{P}
under the assumption \C{B}.

Typically a logic comes with the $\top$, $\bot$ and $\neg$ constants.
They can be defined as follows.

\begin{coq}{name=TrueFalse}{}
Inductive True : Prop := I.
Inductive False : Prop := .
Definition not (A : Prop) := A -> False.
\end{coq}

Remark that to prove \C{True} one has simply to provide \C{I} that has no
argument.  So proving \C{True} is trivial, and as a consequence eliminating it
provides little help (i.e. no extra knowledge is obtained by pattern matching
over \C{I}).  Conversely it is impossible to prove \C{False}, since it has no
constructor, and pattern matching on \C{False} can inhabit any type, since no
branch has to be provided.

\begin{coq}{name=exfalso}{}
Definition exfalso (P : Prop) (f : False) : P :=
  match f with end.  (* no constructors, no branches *)
\end{coq}

The only base predicate we are left to describe is equality.  The reason we
left is as the last one is that it has a tricky nature.  In particular
equality, as we have seen in the previous chapters, is an open notion
in the following sense.  Terms that compute to the same syntactic expression
are considered as equal, and this is true for any program the user may write.
Hence such notion of equality needs to be somewhat primitive, as
\C{match} and \C{fun} are.  One also expects such notion to come
with a substitutivity property: replacing equals by equals must be licit.

The way this internal notion is exposed is via the concept of index
on which an inductive family may vary.

\begin{coq}{name=Eq}{}
Inductive eq (A:Type) (x:A) : A -> Prop := erefl : x = x
where "x = y" := (@eq _ x y) : type_scope.
\end{coq}

This is the first time we see a function type after the \C{:} symbol, line 1.
The \C{eq} type constructor takes three arguments: a type \C{A} and
two terms of that type (the former is named \C{x}).
Hence one can write \C{(a = b)} whenever \C{a} and \C{b}
have the same type.
The \C{erefl} constructor takes no arguments, as \C{I}, but its type
annotation says it can be used to inhabit only the type \C{(x = x)}.
Hence one is able to prove \C{(a = b)} it only when \C{a} and \C{b} are
equal up to computation.  Conversely by eliminating a term
of type \C{(a = b)} one discovers that  \C{a} and \C{b} are
equal and \C{b} can be freely replaced by \C{a}.

\begin{coq}{name=EqInd}{}
Definition eq_ind A (P : A -> Prop) x (px : P x) y (e : x = y) : P y :=
  match e with erefl => px end.
\end{coq}

The notion of equality is one of the most intricate aspects of type
theory and out of the scope of this book.  The interested reader
finds an extensive study of this subject in~\cite{hottbook}.  Later in this
chapter we define and use other inductive families to take advantage
of the ``automatic'' substitution of the implicit equations we see here:
while the type of \C{px} is \C{(P x)} it is accepted as an
inhabitant of \C{(P y)} because inside the \C{match} the term \C{y}
is automatically replaced by \C{x}.

\section{Proof commands}

Since proofs are just terms, one can hardly use the term language to
write long proofs.   The proof commands we have seen so far, like
\C{case:} for example, provide a much more compact syntax for proofs and can be explained in terms of the proof terms they generate behind the scenes.

For example \C{case: n} writes for you a \C{match} expression with the right
shape by looking at the type of \C{n}.  If \C{n} is a natural number then there
are two branches, the one for the \C{S} constructor carries an argument of type
\C{nat}, the other one is for \C{0} and binds no additional term.
The \C{case:} tactic is general enough to work with any inductive data type
and inductive predicate.

The \C{apply:} tactic generates an application.  For example \C{apply: addnC}
generates the term \C{addnC t1 t2} by figuring out the correct values of
\C{t1} and \C{t2}, or opening new goals when this cannot be done, i.e.
if the lemma takes in input proofs, like \C{contraL}.

There is a list of proof commands that are shorthands for \C{apply:}
and is only worth mentioning here briefly. \C{split} proves a conjunction
by applying the \C{conj} constructor, \C{left} and \C{right} prove a
disjunction by applying \C{or\_introl} and \C{or\_intror} respectively.
\C{exist t} proves an existentially quantified formula by providing
the witness \C{t} and, later, a proof that \C{t} validated the predicate.
Finally \C{reflexivity} proves an equality by applying \C{erefl}.

The only primitive constructor that remains without an associated proof command
is \C{(fun .. => ..)}.  Operationally what the $\to_I$ and
$\forall_I$ logical rule do is to introduce into the proof context a
new entry.  So far we either expressed this step at the beginning of proofs
by putting such names just after the name of the lemma being prover, or
just after a \C{case:} or \C{elim:} proof command.  Next section
expands this subject covering the full management of the proof context.

\subsection{Bookkeeping and the stack model}

The presentation we gave so far of proof commands like \C{case: n => [|m]}
is oversimplified.  While \C{case} is indeed the proof command in
charge of performing case analysis the ``\C{: n}'' and ``\C{=> [|m]}''
parts are decorators to prepare the goal and post process the result of
the proof command.  These decorators deal with what we typically call
\emph{bookkeeping}: actions that are necessary in order to obtain readable and
robust proof scripts but that are too frequent to benefit from a move verbose
syntax.  Bookkeeping actions do convoy a lot of information, like where
names are given to assumptions, but also let one deal with annoying details
using a compact, symbolic, language.  Note that all bookkeeping action
correspond to regular, named, proof commands.  It is the use one makes of them
that may be twofold: a case analysis in the middle of a proof may start two
distinct lines of reasoning, and hence it is worth being noted explicitly with
the \C{case} word; conversely de-structuring a pair to obtain the two
components can hardly be a relevant step in a proof, so one may prefer to
perform such bookkeeping action with a symbolic, compact, notation
corresponding to the same \C{case} functionality.

%%%%%%%%%%%%%%%%%%%%%%%%%%%%%%%%%%%%%%%%%%%%%%%%%%%%%%%%%%%%%%%%%
\mcbLEVEL{1}
\mcbsubsubsection{Pulling from the stack}

Lets start with the post processing phase, called \emph{introduction pattern}.
The postfix ``\C{=> ...}'' syntax can be used in conjunction with any proof
command, and it performs a sequence of actions on the first goal assumption or
quantified variable.  With these looking glasses, the goal becomes a
\emph{stack}. Take for example this goal:

\begin{coqout}{name=Stack}{}
========================
forall x, prime x.1 -> odd x.2 -> 2 < x.2 + x.1
\end{coqout}

Before accessing the assumption \C{prime x} one has to name the
bound variable \C{x}, exactly as one can only access a stack from its top.
The execution of \C{=> xy pr\_x odd\_y} is just the composition of
\C{=> xy} with \C{=> pr\_x} and finally \C{=> odd\_y}.  Each action
pulls out of the stack an item and names it.  The \C{move} proof
command is the one that does nothing.  We can use it as a placeholder
for the postfix \C{=>} bookkeeping action.

\begin{coq}{}{width=7cm}
move=> xy pr_x odd_y
\end{coq}
\begin{coqout}{name=Stack1}{width=5cm}
 xy : nat * nat
 pr_x : prime xy.1
 odd_y : odd xy.2
========================
 2 < xy.2 + xy.1
\end{coqout}

Now, en passant, one would like to decompose \C{xy} into its first
and second component.  Instead of the verbose \C{=> xy; case: xy => x y}
one can use the symbolic notation \C{[]} to perform such action.

\begin{coq}{}{width=7cm}
move=> [x y] pr_x odd_y
\end{coq}
\begin{coqout}{name=Stack2}{width=5cm}
 x, y : nat
 pr_x : prime (x,y).1
 odd_y : odd (x,y).2
========================
 2 < (x,y).2 + (x,y).1
\end{coqout}

One can place the \C{/=} switch to force the system to reduce the formulas on
the stack, before introducing them in the context, and obtain:

\begin{coq}{}{width=7cm}
move=> [x y] /= pr_x odd_y
\end{coq}
\begin{coqout}{name=Stack2}{width=5cm}
 x, y : nat
 pr_x : prime x
 odd_y : odd y
========================
 2 < y + x
\end{coqout}

One could also process an assumption trough a lemma.  For example
\C{prime\_gt1} states \C{(prime p -> 1 < p)} for any \C{p}, and we can
use it as a function to obtain a proof of \C{(1 < x)}  from a proof
of \C{(prime x)}.

\begin{coq}{}{width=7cm}
move=> [x y] /= /prime_gt1-x_gt1 odd_y
\end{coq}
\begin{coqout}{name=Stack2}{width=5cm}
 x, y : nat
 x_gt1 : 1 < x
 odd_y : odd y
========================
 2 < y + x
\end{coqout}

The leading \C{/} makes \C{prime\_gt1} do as a function instead of
as a name to be assigned to the top of the stack.  The \C{-} has no effect but
to visually link the function and name assigned to its output. 

One could also examine \C{y}: it can't be \C{0}, since it would contradict
the assumption saying that \C{y} is \C{odd}.

\begin{coq}{}{width=7cm}
move=> [x [//|y]] /= /prime_gt1-x_gt1.
\end{coq}
\begin{coqout}{name=Stack2}{width=5cm}
 x, y : nat
 x_gt1 : 1 < x
 ========================
 ~~ odd y -> 2 < y.+1 + x
\end{coqout}

This time the destruction of \C{y} generates to cases, hence the \C{[ .. | .. ]}
syntax mentioning the two branches.  In the first one, when \C{y} is \C{0},
the \C{//} action solves the goal, by the same trivial means
of the \C{by []} terminator.  In the second branch we name \C{y} the
new variable.

Now, the fact that \C{y} is even is not needed to conclude, so we can discard it using the \C{\_} dummy name.

\begin{coq}{}{}
by move=> [x [//|y]] /= /prime_gt1-x_gt1 _; apply: ltn_addl x_gt1.
\end{coq}

We finally conclude with the \C{apply:} command that here we use
with two arguments: a function and its last argument.

\begin{coq}{}{width=4cm}
About ltn_addl.
\end{coq}
\begin{coqout}{}{width=8cm}
ltn_addl : forall m n p : nat, m < n -> m < p + n
\end{coqout}

\C{apply:} will fill in the blanks between the function (the lemma name)
and the argument provided.  Note that by passing \C{x\_gt1}, the
variable \C{m} picks the value \C{1}.  The conclusion of \C{ltn\_addl}
hence unifies with \C{(2 < y.+1 + x)} because both \C{+} and \C{<} are
defined as programs that compute: addition exposes a \C{.+1} by
reducing to \C{2 < (y+x).+1}, then \C{<}, or better the underlying
\C{<=}, eats a successor from both sides, leading to \C{1 < y + x}
that looks like the conclusion of the lemma we apply.

Here we have shown all possible actions one can perform in an intro
pattern, squeezing the entire proof into a single line.  This has
to be seen both as an opportunity and as a danger: one can easily
make a proof unreadable by performing too many actions in the bookkeeping
operator \C{=>}.  At the same time a trivial sub-proof like this one
should take no more than a line, and in that case one typically
sacrifices readability in favour of compactness: what would you learn by
reading a trivial proof?  Of course,
finding the right balance only comes with experience.

%%%%%%%%%%%%%%%%%%%%%%%%%%%%%%%%%%%%%%%%%%%%%%%%%%%%%%
\mcbLEVEL{1}
\mcbsubsubsection{Working on the stack}

The stack can also be used as a work place.  Indeed there is
no need to pull from the stack all items.  If we take the previous example

\begin{coqout}{name=Stack}{}
========================
forall x, prime x.1 -> odd x.2 -> 2 < x.2 + x.1
\end{coqout}
and we stop just after applying the view, we end up is a valid situation.

\begin{coq}{}{width=6cm}
move=> [x y] /= /prime_gt1.
\end{coq}
\begin{coqout}{name=Stack2}{width=6cm}
 x, y : nat
 ========================
 1 < x -> odd y -> 2 < y + x
\end{coqout}

One can also chain multiple views on the same stack item.

\begin{coq}{}{width=6cm}
move=> [x y] /= /prime_gt1/ltnW.
\end{coq}
\begin{coqout}{name=Stack2}{width=6cm}
 x, y : nat
 ========================
 0 < x -> odd y -> 2 < y + x
\end{coqout}

Two other operations are available on the top stack item: specialization
and substitution.  Let's take the following conjecture.

\begin{coqout}{}{}
========================
(forall n, n * 2 = n + n) -> 6 = 3 + 3
\end{coqout}

The top stack item is a quantified assumption.  To specialize it to, say,
3 one can write as follows:

\begin{coq}{}{width=4cm}
move=> /(_ 3).
\end{coq}
\begin{coqout}{}{width=8cm}
========================
3 * 2 = 3 + 3 -> 6 = 3 + 3
\end{coqout}
The idea behind the notation is that when applies a view to the top
stack item, as in \C{/v}, he is forming the term \C{(v top)}, while
when oen specializes the top assumption as in \C{/(_ x)} he is forming the
term \C{(top x)}.  The \C{_} is a plce holder for the top item.

When the top stack item is an equation one can substitute it using \C{<-}
and \C{->} for right-2-left and left-2-right respectively.

\begin{coq}{}{width=4cm}
move=> /(_ 3) <-.
\end{coq}
\begin{coqout}{}{width=8cm}
========================
6 = 3 * 2
\end{coqout}
In other words, the arrows are just a compact syntax for rewriting,
as in the \C{rewrite} tactic, with the top assumption.

%%%%%%%%%%%%%%%%%%%%%%%%%%%%%%%%%%%%%%%%%%%%%%%%%%%%%%%%%%%%%%%%%
\mcbLEVEL{1}
\mcbsubsubsection{Pushing to the stack}

We have seen how to pull items from the stack to the context.  Now let's see
how to perform the converse via the \C{:} operator.  Such operator decorates
proof commands with actions to be performed before the command is actually run.
Imagine we want to perform case analysis on \C{y} at this stage

\begin{coqout}{name=Stack2}{}
 x, y : nat
 x_gt1 : 1 < x
 odd_y : odd y
 ========================
 2 < y + x
\end{coqout}

Let's dissect the command \C{case: y}.  It is equivalent to
\C{move: y; case.} where \C{move} once again is a place holder,
\C{: y} pushes onto the stack the \C{y} variable and \C{case}
operates on the top of the stack.
Pushing items on the stack is called \emph{discharging}.

Just before running \C{case} the goal looks like this:

\begin{coqout}{name=Stack2}{}
 x : nat
 x_gt1 : 1 < x
 odd_y : odd y
 ========================
 forall y, 2 < y + x
\end{coqout}

Unfortunately the binding for y is needed by the \C{odd\_y}
context item, so \C{case: y} fails.  One has to push on the stack
items in a valid order.  \C{case: y odd\_y} would push \C{odd\_y}
first, then \C{y}, leading to

\begin{coqout}{name=Stack2}{}
 x : nat
 x_gt1 : 1 < x
 ========================
 forall y, odd y -> 2 < y + x
\end{coqout}

Via the execution of \C{case} one obtains:

\begin{coqout}{name=Stack2}{}
2 subgoals

  x : nat
  x_gt1 : 1 < x
  ========================
   odd 0 -> 2 < 0 + x

subgoal 2 is:
 forall n : nat, odd n.+1 -> 2 < n.+1 + x
\end{coqout}

An alternative to discharging \C{odd\_y} would be to clear it, i.e. purge it
from the context.  Listing context entry names inside curly braces has this
effect.  For example \C{case: y \{odd\_y\}}.

One can combine \C{:} and \C{=>} around a proof command, to first prepare the
goal for its execution and finally apply the necessary bookkeeping to the
result.  For example:

\begin{coq}{}{width=6cm}
case: y odd_y => [|y']
\end{coq}
\begin{coqout}{name=Stack2}{width=6cm}
2 subgoals

  x : nat
  x_gt1 : 1 < x
  ========================
   odd 0 -> 2 < 0 + x

subgoal 2 is:
 odd y'.+1 -> 2 < y'.+1 + x
\end{coqout}

At the left of the \C{:} operator one can also put a name for an
equation that links the term at the top of the stack before and
after the tactic execution.  \C{case E: y odd\_y => [|y']} leads to
the two sub goals:

\begin{coqout}{name=Stack2}{width=6cm}
 x, y : nat
 x_gt1 : 1 < x
 E : y = 0
============================
 odd 0 -> 2 < 0 + x
\end{coqout}
\begin{coqout}{name=Stack2}{width=6cm}
 x, y : nat
 x_gt1 : 1 < x
 y' : nat
 E : y = y'.+1
========================
 odd y'.+1 -> 2 < y'.+1 + x
\end{coqout}

\gotcha{: after apply is not the same, it is just to distinguish the apply of Coq from the one of ssr that fills in the blanks}

\gotcha{[] as the first item of \C{=>} never does case, only the branching (not the status quo, but I prefer this exception to the current one)}


\section{Inductive reasoning}

In chapter 1 we have seen the \C{fun} construction of functions and inductive
data types like \C{nat} as well as the pattern matching.  All these
constructions have found a correspondence in the Curry-Howard correspondence.
The only missing piece is recursive programs.  For example
\C{addn} was written by recursion on its first argument, and is a
function taking in input two numbers and producing a third one.
We can write programs by recursion that take in input, among regular  data,
proofs and produce in output other proofs.  Let's look at the
induction principle for natural numbers trough the looking glasses of the
Curry-Howard isomorphism.

\begin{coq}{}{title=Query,width=3.5cm}
About nat_ind.
\end{coq}
\begin{coqout}{}{title=Response,width=9cm}
nat_ind : forall P : nat -> Prop,
  P 0 -> (forall n : nat, P n -> P n.+1) -> forall n : nat, P n
\end{coqout}
\C{nat\_ind} is a program that produces a proof of \C{(P n)} for any \C{n}
proviso a proof for the base case \C{(P 0)}, and a proof
of the inductive step \C{(forall n : nat, P n -> P n.+1)}.
Let's write such program.

\begin{coq}{}{}
Fixpoint nat_ind (P : nat -> Prop)
  (p0 : P 0) (pS : forall n : nat, P n -> P n.+1) n : P n :=
  if n is m.+1 then
    let pm (* : P m *) := nat_ind P p0 pS m in
    pS m pm (* : P m.+1 *)
  else p0.
\end{coq}

The \Coq{} system generates such program when the \C{nat} data type
is defined.  Recall that recursive functions are checked for termination:
trough the lenses of the proofs as programs correspondence this means
that the induction principle just coded is sound.

%%%%%%%%%%%%%%%%%%%%%%%%%%%%%%%%%%%%%%%%%%%%%%%%%%%
\mcbLEVEL{1}
\mcbsubsection{Strong induction}

As an exercise we show how the \C{elim} tactic combined with the bookkeeping
operator \C{:} lets one perform, on the fly, a stronger variant of
induction called ``course of values''.

Claim: every amount of postage that is at least 12 cents
can be made from 4-cent and 5-cent stamps.  The proof in the inductive
step goes as follows.  There are obvious solutions for a postage between
12 and 15 cents, so we can assume it is at least 16 cents.  Since
the postage amount is at least 16, by using a 4-cents stamp we are back
at a postage amount that, by induction, can be obtained as claimed.\hfill$\square$

The trick is that we want to apply the induction hypothesis not
on $n-1$, as usual, but on $n-4$, since we know how to turn a
solution for a stamping amount problem $n$ to one for a problem of
size $n+4$.  The induction hypothesis provided by \C{nat_ind}
is not strong enough.  However we can use the \C{:} operator
to load the goal before performing the induction.
\index[ssr]{\C{set}}

\begin{coq}{}{}
Lemma stamps n : 12 <= n -> exists s4 s5, s4 * 4 + s5 * 5 = n.
Proof.
elim: n {-2}n (leqnn n) =>[|n IHn]; first by case.
do 12! [ case; first by [] ]. (* < 12c *)
case; first by exists 3, 0.   (* 12c = 3 * 4c *)
case; first by exists 2, 1.   (* 13c = 2 * 4c + 1 * 5c *)
case; first by exists 1, 2.   (* 14c = 1 * 4c + 2 * 5c *)
case; first by exists 0, 3.   (* 15c = 3 * 5c *)
move=> m'; set m := _.+1; move=> mn m11.
case: (IHn (m-4) _ isT) => [|s4 [s5 def_m4]].
  by rewrite leq_subLR (leq_trans mn) // addSnnS leq_addl.
by exists s4.+1, s5; rewrite mulSn -addnA def_m4 subnKC.
Qed.
\end{coq}

Just before the induction step the goal is the following one:

\begin{coq}{}{}
forall n m : nat,
  m <= n -> 11 < m -> exists s4 s5 : nat, s4 * 4 + s5 * 5 = m	
\end{coq}
The corresponding induction hypothesis \C{IHn} is:

\begin{coq}{}{}
IHn : forall m : nat,
      m <= n ->
      11 < m -> exists s4 s5 : nat, s4 * 4 + s5 * 5 = m
\end{coq}
Such hypothesis is accessible for each \C{m} that is at least \C{12}
and smaller than \C{n}.

Loading the goal works as follows: first \C{(leqnn n)} is pushed on
the stack, then all occurrences of \C{n} but the second one are push
on the stack obtaining

\begin{coq}{}{}
forall m : nat,
  m <= n -> 11 < m -> exists s4 s5 : nat, s4 * 4 + s5 * 5 = m
\end{coq}
Finally \C{n} is pushed on the stack so that induction is performed
on such quantified variable.

Lines 4, 9 and 10 deserve a few comments.  Line 4 repeats a tactic
12 times.  Line 9 uses the \C{set} proof command to name \C{m} a
term of the form \C{_.+1} where \C{_} is a wildcard. It names
\C{m} the term \C{m'.+16}.  Line 10 passes to the induction
hypothesis \C{(m-4)}, then a placeholder for a missing proof
of \C{(m-4 < n)}, finally a proof that \C{(11 < m-4)}.  The introduction
pattern in line 10 names \C{s4} and \C{s5} the quantities of
4-cents and 5-cents stamps needed to cover the amount of postage \C{(m-4)}.

\section{Axioms of the logic we work with}

Not all valid reasoning principles can be expressed as programs.
For example excluded middle can be proved by a program only when
the predicate is decidable, i.e. when we can write in Coq a program
to \C{bool} that decides it.
Full excluded middle can only be \emph{axiomatized}, i.e. assumed globally.

In \mcbMC{} we stay axiom free.  This makes the library compatible with as many
extra axioms as possible (i.e. not all combinations of axioms are consistent,
hence picking one may rule out others).  Sometimes this is done by Goedel monad,i.e. confining the use of excluded middle inside a box one can open only to prove decidable statements..., sometimes by reworking the math in a weaker setting like Real Closed Fields (weaker than reals, but sufficient for most algebra).
