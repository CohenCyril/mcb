\chapter{Linear algebra}
\label{linalg:smith}

The library covers linear algebra in finite dimension in two distinct
and almost independent parts. The first part provides a notion of
abstract finite dimension vector spaces on top of the algebraic
hierarchy. The second part represents vectors, families of vectors,
spaces and linear maps using the same matrix operations, and with a
maximal sharing of operations.

Warning: If you are going to formalize linear algebra, do not choose
lightly between the two parts. It is a difficult thing to understand
which level of abstraction to choose. Often the sensible thing to do
is to use mxalgebra, as soon as you use a matrix somewhere in your
development.

\section{Vector spaces}
\label{sec:vector-spaces}

\paragraph{Contents}

\begin{itemize}
\item We provide an abstract interface for vector spaces, which
  inherits from the interface from left modules from the algebraic
  hierarchy, together with the axiom \C{Vector.axiom_def} which states
  an isomorphism between the vector space and $R^n$ for some $n$. This
  axiom should be read as: ``any vector is uniquely determined by its
  coordinates in a fixed basis'' All the theory of this interface is
  described in \lib{algebra.vector}.
\end{itemize}

\paragraph{Formalization choices}

The library provides reasoning on vector spaces (addition, complement,
intersection) and linear maps (identity, composition and inverse) as a
primitive objects, with a few bridges with elements (taking the space
spanned by a family of vectors, taking a basis or picking a non-zero
element). \emph{There is no bridge from and to matrices at this
  stage}, so if you want to deal with matrix representations of linear
maps, you would rather use the other linear algebra library.

The incomplete basis theorem is not provided as a theorem in the
library but is rather viewed as a composition of operations: first
take the complement space and the get its basis.

\begin{coq}{}{width=8cm}
Theorem incomplete_basis_theorem (F : fieldType) (E : vectType F) (v : seq E) :
  free v -> exists v', basis_of fullv (v ++ v').
Proof.
move=> v_free; exists (vbasis <<v>>^C); rewrite /basis_of ?subvf /=.
rewrite cat_free v_free (basis_free (vbasisP _)) /= span_cat.
rewrite (span_basis (vbasisP _)) directv_addE !directvE capv_compl !eqxx /=.
by rewrite addv_complf eqxx.
Qed.
\end{coq}

This proof is quite long, however, in practice the incomplete basis
theorem is never used as such, but only in pieces.

Galois theory is based on this library but is in another chapter.

\section{Matrix based linear algebra}
\label{sec:matrix-based-linear}

\paragraph{Contents}

\begin{itemize}
\item The type of matrices and basic operation on them (\textit{i.e.}
  operations independent from linear algebra) are defined in the
  library \lib{algebra.matrix}. 

  Things in matrix.v that are important for mxalgebra: blocks,
  non-zero dimension of matrices, \C{lin_mx}.

\item The library \lib{algebra.mxalgebra} overloads matrix product
  which means at the same time: composition of linear maps,
  application of maps and image of space by a map. This simplifies the
  theory as there is less lemmas to write.

\end{itemize}


\section{Comparison between vector spaces and matrix algebra}
\label{sec:comp-betw-vect}

The library \lib{algebra.vector} has the same primitives as mxalgebra
but with additional casts to go from one type to the other, that were
identified in mxalgebra.

A usual treatment of linear algebra would require the use of mxalgebra
and not vector. Indeed, going back and forth between matrix
representation and linear maps or spaces is only possible for
mxalgebra.



\newcommand{\idmat}{\ensuremath{\mathbb{I}}}

\section{Detailed Contents}
In the next revision of the book?

\paragraph{Matrices}

The starting point for linear algebra is the library
\lib{algebra.matrix}, which contains the basic definition of matrices
(as a table of number represented by finite functions), and the
following ``basic'' operations that are independent form the type:
\begin{itemize}
\item building a matrix from its coefficients using big notations
  \C{$\backslash$matrix_(i < m, j < n) c_ i j}, \C{$\backslash$row_(j < n) c_ j}, \C{$\backslash$col_(i <
    m) c_ i} and \C{$\backslash$matrix_(i < m) row_ i}.
\item taking a coefficient from the matrix by a simple function
  application \C{M i j}
\item mapping (as in list maps) the coefficients of a matrix
  \C{map_mx}.
\item transposing a matrix \C{_^T}
\item taking (\C{row}, \C{col}) or dropping a line or a column
  (\C{row'}, \C{col'})
\item exchanging two rows \C{xrow} or two columns \C{xcol}, or
  permuting rows \C{row_perm} or columns \C{col_perm} using a
  permutation from the library \lib{fingroup.perm}
\item building block matrices
\end{itemize}

If the base type has more structure, then one can provide more
structure on matrices as well. In particular matrix type inherit the
\C{eqType}, \C{choiceType}, \C{countType}, \C{finType} and
\C{zmodType}. When the base type is a ring, we also have:
\begin{itemize}
\item The matrix multiplication \C{_ * _}
\item Scalar matrices $c\cdot \idmat$ (\C{c\%:M}) and diagonal matrices
  \C{diag_mx d}.
\item Identity \C{1:\%M} and partial identity matrices \C{pid_mx},
  \C{copid_mx}.
\item Elementary matrices
  $E_{i_0,j_0} = \left(\delta_{i,i_0}\cdot\delta_{j,j_0}\right)_{i,j}$
  (\C{delta_mx}).
\item Permutation matrices \C{tperm_mx}, \C{perm_mx}
\item Trace \C{$\backslash$tr} and determinate \C{$\backslash$det}
\item The cofactor \C{cofactor} and the adjoint matrix
  \C{$\backslash$adj}
\item Invertibility test \C{unitmx} and inversion of a matrix
  \C{invmx}
\end{itemize}

\paragraph{linear maps and spaces as matrices}

The library \lib{algebra.mxalgebra} provides an interpretation of row
matrices as vectors and matrices both as linear maps and as the
subspace of $R^n$ generated by the row vectors of matrices. Note that
this particular choice, \textit{i.e.} representing vectors as rows and
not as columns, implies the use of the english convention instead of
the french convention for application of a linear map to a
vector, since we write: $v \cdot f$ for the application~$f(v)$ of~$f$
to~$v$.

Hence, a matrix $M$ can be, depending on the context, read either as a
linear map or a space. Operations on matrices seen as linear
functions have their notations in the ring scope \C{\%R}, while
operations as spaces have their notations in the matrix space scope
\C{\%MS}.

Basic operations on matrices seen as spaces:
\begin{itemize}
\item The rank of the matrix is also the dimension of the space
  generated by its row vectors, \C{$\backslash$rank}
\item Test whether the family of row vectors of a matrix is free or
  full (\C{row_free}, \C{row_full}
\item Computation of the basis of the space
\item Test
\end{itemize}

Basic operations on matrices seen as linear maps:
\begin{itemize}
\item The kernel (\C{kermx}) of the map, represented as a matrix
\item Test if a map has a given eigenvalue, and the eigenspace
  associated with it, represented as a matrix
\end{itemize}


\paragraph{Advanced features on Matrices}

There are also more advanced operations, that are unrelated to the
math:
\begin{itemize}
\item changing the type of the matrix using equalities on its
  dimensions (\C{castmx}) or forcing a matrix to have some dimensions
  that we know to be equal to the original ones (\C{conform_mx}).
\item 
\end{itemize}

\paragraph{Matrices}

\section{Formalization Choices}

\section{Notations and conventions}

\section{Formalization trick}

\C{mxdirect}

