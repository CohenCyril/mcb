\chapter{Linear algebra}
\label{linalg:smith}

TODO:

\paragraph{Contents}
Only finite dimension covered

\begin{enumerate}
\item Begins with matrix : array of elements of a type with more or
  less structure.  
\item Matrices are the basis of finite dimensional
  algebra (mxalgebra).
\item   And then an abstraction to cover vector spaces
  in a more traditional way.
\end{enumerate}

Warning: Do not choose lightly. It is a difficult thing to understand
which level of abstraction to choose. Often the sensible thing to do
is to use mxalgebra. Point where to find which one to use more
precisely.

Indeed mxalgebra overloads matrix product which means at the same
time: composition of linear maps, application of maps and image of
space by a map. This simplifies the theory as there is less lemmas to
write.

\paragraph{Not contents anymore}
\begin{itemize}
\item Examples. How to use the language of mxalgebra compared to
  vector spaces.

\item   A usual treatment of linear algebra would require the use of
  mxalgebra and not vector. Indeed, going back and forth between
  matrix representation and linear maps or spaces is only possible for
  mxalgebra.

\item   Vector spaces is for Galois theory.

\item   Representation theory is in another chapter.

\item   How to use mxalgebra and vector spaces.

\item Incomplete basis theorem: Not a theorem in the library but provided
  in kits.

\item   Take three folklore statements and show the bricks. Be in a goal
  where it is needed, and show how to use the library.


\item   vector has the same primitives as mxalgebra but with additional
  casts to go from one type to the other, that were identified in
  mxalgebra.


\item   Independent chapters for vector and mxalgebra.

\item   Things in matrix.v that are important for mxalgebra: blocks,
  non-zero dimension of matrices, \C{lin_mx}.

\end{itemize}
END OF TODO








The library covers linear algebra in finite dimension in two distinct
parts. The first part is using matrices to represent everything and
the second is building vector spaces on top of the algebraic
hierarchy.



\newcommand{\idmat}{\ensuremath{\mathbb{I}}}

\section{Contents}

\paragraph{Matrices}

The starting point for linear algebra is the library
\lib{algebra.matrix}, which contains the basic definition of matrices
(as a table of number represented by finite functions), and the
following ``basic'' operations that are independent form the type:
\begin{itemize}
\item building a matrix from its coefficients using big notations
  \C{$\backslash$matrix_(i < m, j < n) c_ i j}, \C{$\backslash$row_(j < n) c_ j}, \C{$\backslash$col_(i <
    m) c_ i} and \C{$\backslash$matrix_(i < m) row_ i}.
\item taking a coefficient from the matrix by a simple function
  application \C{M i j}
\item mapping (as in list maps) the coefficients of a matrix
  \C{map_mx}.
\item transposing a matrix \C{_^T}
\item taking (\C{row}, \C{col}) or dropping a line or a column
  (\C{row'}, \C{col'})
\item exchanging two rows \C{xrow} or two columns \C{xcol}, or
  permuting rows \C{row_perm} or columns \C{col_perm} using a
  permutation from the library \lib{fingroup.perm}
\item building block matrices
\end{itemize}

If the base type has more structure, then one can provide more
structure on matrices as well. In particular matrix type inherit the
\C{eqType}, \C{choiceType}, \C{countType}, \C{finType} and
\C{zmodType}. When the base type is a ring, we also have:
\begin{itemize}
\item The matrix multiplication \C{_ * _}
\item Scalar matrices $c\cdot \idmat$ (\C{c\%:M}) and diagonal matrices
  \C{diag_mx d}.
\item Identity \C{1:\%M} and partial identity matrices \C{pid_mx},
  \C{copid_mx}.
\item Elementary matrices
  $E_{i_0,j_0} = \left(\delta_{i,i_0}\cdot\delta_{j,j_0}\right)_{i,j}$
  (\C{delta_mx}).
\item Permutation matrices \C{tperm_mx}, \C{perm_mx}
\item Trace \C{$\backslash$tr} and determinate \C{$\backslash$det}
\item The cofactor \C{cofactor} and the adjoint matrix
  \C{$\backslash$adj}
\item Invertibility test \C{unitmx} and inversion of a matrix
  \C{invmx}
\end{itemize}

\paragraph{linear maps and spaces as matrices}

The library \lib{algebra.mxalgebra} provides an interpretation of row
matrices as vectors and matrices both as linear maps and as the
subspace of $R^n$ generated by the row vectors of matrices. Note that
this particular choice, \textit{i.e.} representing vectors as rows and
not as columns, implies the use of the english convention instead of
the french convention for application of a linear map to a
vector, since we write: $v \cdot f$ for the application~$f(v)$ of~$f$
to~$v$.

Hence, a matrix $M$ can be, depending on the context, read either as a
linear map or a space. Operations on matrices seen as linear
functions have their notations in the ring scope \C{\%R}, while
operations as spaces have their notations in the matrix space scope
\C{\%MS}.

Basic operations on matrices seen as spaces:
\begin{itemize}
\item The rank of the matrix is also the dimension of the space
  generated by its row vectors, \C{$\backslash$rank}
\item Test whether the family of row vectors of a matrix is free or
  full (\C{row_free}, \C{row_full}
\item Computation of the basis of the space
\item Test
\end{itemize}

Basic operations on matrices seen as linear maps:
\begin{itemize}
\item The kernel (\C{kermx}) of the map, represented as a matrix
\item Test if a map has a given eigenvalue, and the eigenspace
  associated with it, represented as a matrix
\end{itemize}


\paragraph{Advanced features on Matrices}

There are also more advanced operations, that are unrelated to the
math:
\begin{itemize}
\item changing the type of the matrix using equalities on its
  dimensions (\C{castmx}) or forcing a matrix to have some dimensions
  that we know to be equal to the original ones (\C{conform_mx}).
\item 
\end{itemize}

\paragraph{Matrices}

\section{Formalization Choices}

\section{Notations and conventions}

\section{Formalization trick}

\C{mxdirect}

