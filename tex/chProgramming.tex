\chapter{Computational definitions \\ The syntax of terms \\[2ex]\Large\itshape Defining concepts by writing programs} 

Find a more catchy title? The motivation is: how to define things:
objects, operations, (boolean) relations.

Theoretical content:
\begin{itemize}
\item Functions with simple types ($\lambda$, $\rightarrow$): functions are the primitive concept
\item Inductive {\bf datas}
\item Programs by case analysis and recursion
\item Compute (beta, delta, iota) example of beta (that helps later on to explain a predicate for elim principles
\item Polymorphic datatypes (introduce $\Pi$, and its \Coq{} notation
  \C{forall}, for quantification over datatypes only)
\item Everyone has a type : \C{0 : nat : Type : Type}, types avoid confusion
\end{itemize}
This is more or less a standard introduction to (a flavor of)
functional programming, with two possible difficulties:
\begin{itemize}
\item Be precise but not too technical (e.g. on inductive types)
\item Find a line of speech which does not bore/discourage
  mathematicians.
\item Somehow the syntax of (this fragment of the) terms should made
  be clear and precise.
\end{itemize}

\Coq{} commands and features:
\begin{itemize}
\item Implicit arguments (only to go from system F to ML), \C{@}
\item Sections and its discharging, implicit types
\end{itemize}

Comparison with other approaches:
\begin{itemize}
\item Compare an axiomatic, equational presentation of arithmetic to
  its formalization as an inductive type with functions that
  compute. At this stage, where we do not have conversion yet, we
  cannot say much about the proofs and may be just point out that
  computation in \Coq{} is geared toward the reduction of functional
  programs.
\end{itemize}

\Coq{} types introduced:
\begin{itemize}
\item \C{bool, nat, seq, option, prod}
\end{itemize}

Programs presented in detailed examples/exercises:
\begin{itemize}
\item Elementary programs on \C{option}: \C{odflt, obind,}\dots
\item Elementary programs on \C{seq} (without the \C{eqType}):
  \C{size, map, iota,...}
\item Comparison functions on \C{bool, nat}
\item Comparison functions on containers, taking the comparison
  function on the type of stored elements in argument (mind the
  higher-order)
\item Boolean connectives, arithmetic operations on \C{nat}
\item Euclidean division, computation of prime factors, examples from
  elementary number theory
\item Examples of G{\"o}del-style encoding from the {\tt choice.v} library.
\end{itemize}
If possible give a few context to the exercises, in order not to bore
the reader not familiar with programming. For instance do not say that
you encode sequences of nats in nats but give a few hints about the
use of G{\"o}del encoding.

