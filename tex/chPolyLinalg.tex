\chapter{Polynomials, Linear algebra}

2-stage presentation: interface plus explicit. Expansion of
Georges'ITP paper. Here is one of the main application of the choice
operator (complement a base).


\section{Polynomials}

The part of the library covering univariate is composed of two main files
\C{poly} and \C{polydiv}. It contains the basic operations to manipulate
polynomials: usual ring operations, horner evaluation, composition, derative, division, remainder, gcd.

\paragraph{Basic formalization choices} 
The type of polynomials is \C{\{poly R\}}. It is parametrized over a type \C{R}
that is equipped with a ring type structure.
A polynomial is represented by a sequence of coefficients whose last element is non-zero.
This is why there is an automatic coercion from a polynomial to its associated sequence.
Note that there is no explicit definition of the degree of a polynomial in the library, 
the size of  its sequence is used instead. This discrepancy with usual mathematical practice can
be seen in theorems like the one on the degree of the product of two 
polynomials 

\begin{coq}{}{}
Lemma size_mul_leq p q : size (p * q) <= (size p + size q).-1.
\end{coq}
\paragraph{Formalization trick: nested polynomial}
One of the  main result that is derived in the file \C{poly} is that polynomials
form a ring. This means, in particular, that one can create polynomials 
whose coefficient are polynomials \C{\{poly \{poly R\}\}}. This gives a direct way to 
represent bi-variate polynomials. One could go a step further and performs this application
recursively.
This way of getting multivariate polynomials may be interesting for some applications
but is not supported by the library.

\paragraph{Formalization trick: division and pseudo-division}
Division comes in two flavors in the library. A first definition \C{rdivp} applies on an arbitrary ring. 
It consists on performing a pseudo-divisor where the dividend can be multiplied by a scaling factor.
A second one \C{divp} on which is attached the notation \C{p \%/ q} requires a stronger assumption: the domain on which polynomials are built must be
an integral domain type. In this last version, the division tests if the leading coefficient of the divisor
is a unit. If it is the case it performs the expected inversion. This second division when applied on a field gives
the usual equality.

\begin{coq}{}{}
Lemma divp_eq p q : p = (p %/ q) * q + (p %% q).
\end{coq}

\noindent
The library is organized in such a way to provide appropriate theorems for the different domains on top of which
polynomials can be built. Modules are used to regroup
these theorems . For example, here are the various  versions of \C{div_eq}.

\begin{coq}{}{width=12.5cm}
Lemma Pdiv.Field.divp_eq (F : fieldType) (p q : {poly F}) :
  p = (p %/ q * q + p %% q)%R
Lemma Pdiv.Idomain.divp_eq (R : idomainType) (p q : {poly R}) :
  (lead_coef q ^+ scalp (R:=R) p q *: p)%R = (p %/ q * q + p %% q)%R
Lemma Pdiv.IdomainMonic.divp_eq (R : idomainType) (q : {poly R}) :
  q \is monic -> forall p : {poly R}, p = (p %/ q * q + p %% q)%R
Lemma Pdiv.IdomainUnit.divp_eq (R : idomainType) (d : {poly R}) :
  lead_coef d \is a GRing.unit -> forall p: {poly R}, p = (p %/ d * d + p %% d)%R
Lemma Pdiv.WeakIdomain.divp_eq (R : idomainType) (p q : {poly R}) :
  (lead_coef q ^+ scalp (R:=R) p q *: p)%R = (p %/ q * q + p %% q)%R
\end{coq}

\noindent
When loading the  \C{polydiv} file, it is the \C{Pdiv.Field} module that is opened by default.

\section{TODO: Linear algebra}
\C{mxalgebra}

\begin{coq}{}{}
Fixpoint Gaussian_elimination {m n} : 'M_(m, n) -> 'M_m * 'M_n * nat :=
  match m, n with
  | _.+1, _.+1 => fun A : 'M_(1 + _, 1 + _) =>
    if [pick ij | A ij.1 ij.2 != 0] is Some (i, j) then
      let a := A i j in let A1 := xrow i 0 (xcol j 0 A) in
      let u := ursubmx A1 in let v := a^-1 *: dlsubmx A1 in
      let: (L, U, r) := Gaussian_elimination (drsubmx A1 - v *m u) in
      (xrow i 0 (block_mx 1 0 v L), xcol j 0 (block_mx a%:M u 0 U), r.+1)
    else (1%:M, 1%:M, 0%N)
  | _, _ => fun _ => (1%:M, 1%:M, 0%N)
  end.
\end{coq}

Gaussian elimination of a matrix
\C{A} is a triple \C{(L, U, r)} with \C{L} a column permutation of a
lower triangular invertible matrix, \C{U} a row permutation of an upper
triangular invertible matrix, and \C{r} the rank of \C{A}, all
satisfying the identity \C{(L *m pid_mx r *m U = A)}.
\C{(pid_mx r)} is a non necessarily square matrix that has \C{1} in \C{r}
first coefficients on the diagonal and \C{0} elsewhere.

Its main difference with respect to the standard algorithm
is the \emph{double} pivoting \C{[pick ij | A ij.1 ij.2 != 0]}, that
looks for a non-null coefficient in the entire matrix.  By bringing
this element in the top-left cornet iteratively one also computes the rank
of the matrix.

rank theory, incomplete basis thm

subspace equality and choice type

direct sum

\cite{gonthier:hal-00805966}

\section{TODO: Matrices and polynomials}
\C{mxpoly}

Cayley Hamilton, Sylvester matrix, characteristic polynomial
