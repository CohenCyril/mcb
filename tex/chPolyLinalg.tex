\chapter{Polynomials, Linear algebra}

2-stage presentation: interface plus explicit. Expansion of
Georges'ITP paper. Here is one of the main application of the choice
operator (complement a base).


\section{TODO: Plynomials}
\C{poly} and \C{polydiv}

\section{TODO: Linear algebra}
\C{mxalgebra}

\begin{coq}{}{}
Fixpoint Gaussian_elimination {m n} : 'M_(m, n) -> 'M_m * 'M_n * nat :=
  match m, n with
  | _.+1, _.+1 => fun A : 'M_(1 + _, 1 + _) =>
    if [pick ij | A ij.1 ij.2 != 0] is Some (i, j) then
      let a := A i j in let A1 := xrow i 0 (xcol j 0 A) in
      let u := ursubmx A1 in let v := a^-1 *: dlsubmx A1 in
      let: (L, U, r) := Gaussian_elimination (drsubmx A1 - v *m u) in
      (xrow i 0 (block_mx 1 0 v L), xcol j 0 (block_mx a%:M u 0 U), r.+1)
    else (1%:M, 1%:M, 0%N)
  | _, _ => fun _ => (1%:M, 1%:M, 0%N)
  end.
\end{coq}

Gaussian elimination of a matrix
\C{A} is a triple \C{(L, U, r)} with \C{L} a column permutation of a
lower triangular invertible matrix, \C{U} a row permutation of an upper
triangular invertible matrix, and \C{r} the rank of \C{A}, all
satisfying the identity \C{(L *m pid_mx r *m U = A)}.
\C{(pid_mx r)} is a non necessarily square matrix that has \C{1} in \C{r}
first coefficients on the diagonal and \C{0} elsewhere.

Its main difference with respect to the standard algorithm
is the \emph{double} pivoting \C{[pick ij | A ij.1 ij.2 != 0]}, that
looks for a non-null coefficient in the entire matrix.  By bringing
this element in the top-left cornet iteratively one also computes the rank
of the matrix.

rank theory, incomplete basis thm

subspace equality and choice type

direct sum

\cite{gonthier:hal-00805966}

\section{TODO: Matrices and polynomials}
\C{mxpoly}

Cayley Hamilton, Sylvester matrix, characteristic polynomial
