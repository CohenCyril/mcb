\chapter{Numbers}

What are the numbers available in the library? How to use them, casts
between types... Non trivial point in the formalisation:
axiomatization, order is defined from norm, partial orders (in
particular for complex numbers).


The library provides integers, rational numbers, real algebaric
numbers and complex numbers.

The numeric library does not cover natural numbers and there is no
semi-ring structure in the algebraic hierarchy. The theory of natural
numbers is hence kept separate in \C{ssrnat}. Nevertheless, the
theorems that accomplish the same result use same naming
convention. For exemple, \C{addnC} and \C{addrC} express the
commutativity of addition in respectively \C{nat} and an arbitrary
ring.

Requires: ssralg, nat
Provides: ssrnum, more nat?, int, rat, rcf, complex and algC.

\begin{enumerate}
\item int and rat (Beginner)
\item The choice of interfaces (Inter)
\item Content of ssrnum (Advanced)
\end{enumerate}

\section{Integers and rationals}
\label{sec:integers-rationals}

Integers (resp. rationals) are provided an ordered integral domain
structure (resp. ordered field structure) through the
generic mechanism of overloading described in~\ref{ch:hierarchy}.

In particular, the following symbols are available on both structures:
\C{0}, \C{1}, \C{+}, $\ldots$.

The theory of these symbols is also generic and covered in the ssrnum
library.

What is not covered by the ssrnum library is specific operations.

\subsection{Integers}
\label{sec:integers}

On integers we provide a specific injection from natural numbers
\C{\%:Z} 

\begin{coq}{name=about_Posz}{width=5cm,title=injection from $\mathbb{N}$
  to $\mathbb{Z}$}
Locate "%:Z".
About Posz.
\end{coq}
\coqrun{name=about_Posz_run}{about_Posz}
\begin{coqout}{run=about_Posz_run}{title=Response,width=7cm}
Notation            Scope     
"n %:Z" := Posz n    : int_scope ...
intmul : forall R : zmodType, R -> int -> R
Argument scope is [nat_scope] ...
\end{coqout}

Every integer can be constructed from an injection from natural
numbers to integers and arithmetic operations (in particular,
negation).  For example $-42$ is constructed by negating the injection
of $42$: \C{- 42\%:Z}.

In fact, integers are defined in the following way.

\begin{coq}{name=int}{width=12cm,title=Definition of $\mathbb{Z}$}
CoInductive int : Set := Posz of nat | Negz of nat.
\end{coq}

But one should never use the specificities of this implementation, in
particular,
\begin{itemize}
\item Negz should not be used, but rather \C{-_\%:Z},
\item one should not match on an integer, but rather use comparison
  operations \C{_ <= _} followed by the use of absolute
  value. Additionally, the numeric library provides views to reason
  easily on such case distinctions: \C{lerP}, \C{ltrP}, \C{ltrgtP},
  \C{ger0P}, \C{ler0P}, \C{ltrgt0P}.
\end{itemize}


This injection is in fact a coercion, so that when an integer is
expected, one can provide a natural number.

Additionally we provide extension of the operations from the theory of
$\mathbb{Z}-$modules which external operations (\C{*+}, \C{*-},
\C{\%:R}) were taking natural numbers as arguments. We now provide
their generalization \C{*~} and \C{\%:~R}, which are convertible to the
former operations when applied to a natural number.
Indeed \C{x *~ m} is convertible to \C{x *+ m} when m is a natural
number, \C{x *~ (- m.+1\%:Z)} is convertible to \C{x *- m.+1} and
\C{m\%:~R} is convertible to {m\%:R}.

There is a similar extension for powers of ring elements \C{^} is the
generalization of both \C{^+} and \C{^-}. Indeed \C{x ^ m} is
convertible to \C{x ^+ m} when m is a natural number, %
\C{x ^ (- m.+1\%:Z)} is convertible to \C{x ^- m.+1}.


The library ssrnum has a generic sign operation \C{Num.sg : R -> R} which
returns \C{-1}, \C{0} or \C{1} in \C{R} as a sign. The library on
integers provides an additional sign function \C{sgz : R -> int} which
gives the same result in \C{int} instead of \C{R}. \C{Num.sg} and
\C{sgz} are convertible on \C{int}, but in general \C{Num.sg x} and
\C{(sgz x)\%:~R} are not convertible, however they are provably equal
through the lemma~\C{sgrEz}.

We provide the absolute value function \C{absz} denoted with \C{`|n|}
which takes an integer to a natural number. Additionally \C{`|m - n|}
has a special treatment in order for it to be used without the integer
library. This means that one can just import the module \C{IntDist} and
take advantage of distance between natural numbers without any
knowledge about numbers. \C{IntDist} is instantiated using numeric
libraries, but can be used without, since the interface does not
expose any integer.

\subsection{Rationals}
\label{sec:rationals-1}

On integers we provide a specific injection from integers \C{\%:Q},
which is \textbf{not} a coercion, because then it would be ambiguous
to the reader whether \C{m + n} should be read \C{m\%:Q + n\%:Q} or
\C{(m + n)\%:Q}.

Furthermore, \C{m\%:Q} is an alias for \C{m\%:~R} specialized to
rationals, and its theory is hence shared through the algebraic
hierarchy.

We also provide an embedding \C{ratr} of rationals into an arbitrary
unit ring (we need the existence of an inverse, even when it is not
well defined). \C{ratr} is a morphism, as expected, when the target is a
numeric field (which guarantees both that the characteristic is zero
and the existence of inverse for non zero elements). Additionally, we
provide that 

Rationals are defined like this:

\vspace{1em}

\begin{coq}{name=rat}{width=12cm,title=Definition of $\mathbb{Q}$}
Record rat : Set := Rat {
  valq : (int * int) ;
  _ : (0 < valq.2) && coprime `|valq.1| `|valq.2|
}.
\end{coq}

But one should never use the specificities of this implementation, in
particular,
\begin{itemize}
\item one should not use the constructor \C{Rat}, but instead the
  generic division operation, composed with injections from integers:
  \C{_\%:Q / _\%:Q}.
\item one should not pattern match on \C{rat} but use the operations
  \C{numq} and \C{denq} which provide the numerator and denominator of
  the reduced representative.
\end{itemize}

Additionally, we provide the subring of integers \C{Qint} and the
sub-semi-ring of natural numbers \C{Qnat} embedded into rationals as
predicates. These predicates have canonical structures drawn from the
algebraic hierarchy. So one can use the generic theory of substructure
predicates of the algebraic hierarchy on them.


\subsection{The numeric hierarchy}
\label{sec:numeric-hierarchy}

The numeric hierarchy was specifically designed to cover the following
instances: $\mathbb{Z}$, $\mathbb{Q}$, real algebaric numbers and
complex numbers. Hence, we did not provide a general theory of ordered
structure, but only of structures with a (potentially partial) order
(\C{_ <= _} or \C{_ < _}) that is tied to a norm operation (\C{`|_|}
that encompass both absolute value and complex modulus). This is why
the axiomatic of these numbers can be instantiated though any of the
following operations and their axiomatic: large ordering, strict
ordering and norm.

The link between large and strict ordering is obvious and the link
between ordering and norm is \C{(0 <= x) = (`|x| == x)}.

Remark that we provide a partial order on complex numbers in order to
be able to compare complex numbers that happen to be real. In
particular complex number comparable to zero (or to any integer) are
real. Some non-real can also comparable, but we advise not to rely on
this specificity which a convention

NOT TRUE YET: Additionnally, we provide an interface to use classical real and
complex number

% Local Variables:
% ispell-dictionary: english
% End:

