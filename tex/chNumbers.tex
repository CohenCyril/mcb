\chapter{Numbers}

The library contains various data structure for ``numbers''
starting from natural numbers, integers, rationals, Cauchy reals, as well as
related constructions as modular integers, algebraic real numbers and algebraic
complex numbers.

\paragraph{Contents}
\begin{itemize}

\item
Natural numbers $\mathbb{N}$ play a special role in the \mcbMC{} library as they
are such a basic notion that many other concept needs it (eg sequences, ...).
The definition of elementary arithmetic operations on natural numbers
is given and studied in \lib{ssreflect.ssrnat}, except for division
that is given in \lib{ssreflect.div}.  Elementary properties of
prime numbers and factorization in primes are proved in
\lib{ssreflect.prime}.  Falling factorial and binomial coefficients are defined
and studied in \lib{ssreflect.binomial}.

\item
The data structure of integers $\mathbb{Z}$ is defined in \lib{algebra.ssrint}
as well as the operations that equip it with a structure of
ordered ring. Library \lib{algebra.intdiv} provides various
results about integer divisibility.  This library also provides
results on polynomials and matrices with integer coefficients
such as properties of the content of a polynomial or the
existence of the Smith normal form for integer matrices~\ref{linalg:smith}.
Library \lib{algebra.zmodp} defines arithmetic modulo a natural number $n$
and the field structure of $\mathbb{Z}/\mathbb{Z}_p$ for $p$ a prime.

\item
The library \lib{algebra.rat} defined the ordered Archimedean ring of
rational numbers $\mathbb{Q}$.

\item
Libraries \lib{field.algC} and \lib{field.algebraic\_fundamentals}
provide a construction of algebraic closures and in particular define
real and complex algebraic numbers.

\item
Library \lib{real\_closed.cauchyreals} defines Cauchy real numbers.

\item
The generic theory of ... and order are in \lib{algebra.ssrnum}
and \lib{algebra.ssralg}.

\end{itemize}

\paragraph{Formalization Choices}

Binomial coefficients are only defined over natural numbers.

Division on integers (choice about the sign, no unique solution).

zmodp is not a quotient of Z but the finite type ordinals equipped with
arith ops.

cauchy reals have no decideable equality, not an eqtype, unless you axiom
but then the setoid presentation is stupid.

minor: int = pos nat + neg\_or\_zero nat (you get only 2 cases in proofs)

minor: rat is simple subtype

algebraic closure is nonstandard/clever.

sg/sgz x * x = abs x??

\paragraph{Notations and conventions}

Convention: belonging to a ``subring'' is written \C{\\is a}
and comes with lemmas.

ring and orderedring notations.

``r'' in lemma names, slide 10 of
\url{http://www-sop.inria.fr/marelle/math-comp-tut-16/MathCompWS/lesson4.html}

\paragraph{Formalization trick: ...}

spec for 3-way case analysis of int.

pi.-predicate (nat\_pred)

expn x 1 ~ x
expn x 2 ~ x * x

n < m = n.+1 <= m hence 1 leq\_trans + less lemmas + more conv

nattrec + passerelle

ring

Num of binint

case: (ex\_minnP exP) => m pm m\_minP.

leqif n m C   =   n <= m /\ (n == m = C)
