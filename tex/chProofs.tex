\Chapter{Statements and proofs}{First steps in formal proofs}

The scope of this chapter is to explain how one states a lemma,
proves it, finds it.  Focus on simple proof commands, conventions
and boolean reflextion.  Only later we explain what a proof really is
in Coq.

\mcbLEARN{Lemma, rewrite}
\mcbREQUIRE{}
\mcbPROVIDE{eq, rewrite, pattern, quantifiers, is-true, simpl, naming convention}
\mcbLEVEL{1}
\mcbsection{Equality}

Among the many predicates available in \Coq{},
equality plays a privileged role in the \mcbMC{} library.

In this section we consider it as a primitive notion and we focus on its use to
state and prove theorems.  The equality predicate is named \C{eq} and
the infix notation \C{=} is attached to it.

\begin{coq}{width=6cm,title=Equality predicate}
Check @eq.
Check 3 = 3.
Check true = false || true.
\end{coq}
\begin{coqout}{title=Response,width=6cm}
eq : forall A : Type, A -> A -> Prop
3 = 3 : Prop
true = false || true : Prop
\end{coqout}

The first thing to notice is that the equality predicate is polymorphic.
Exactly as we have seen in the previous chapter then \C{forall} quantifier
is used to make the variable \C{A} range over types.  By applying  
\C{eq} to a type and to two extra arguments one obtains a term
of type \C{Prop}, i.e. a proposition.  A proposition is something
one typically proves.

\begin{coq}{width=4cm,title=Equality predicate}
Check 3 = 4.
Check 3 = [:: 3].
$~$
\end{coq}
\begin{coqout}{title=Response,width=8cm}
3 = 4 : Prop
Error: The term "[:: 3]" has type "seq nat"
 while it is expected to have type "nat".
\end{coqout}

It is important to remark that the \Coq{} system checks the well typedness of
statements, i.e. that the two hand sides of an equality have the same type, but
not the truth.  It is the task of the user to prove it.

\subsection{Stating and proving an equality}

We are ready to state our first lemma by employing the \C{Lemma} keyword,
followed by the name we chose for the lemma and the lemma statement.
The \C{Proof} keywords denotes the beginning of the proof.

\begin{coq}{width=6cm,title=Starting a proof}
Lemma my_first_lemma : 3 = 3.
Proof.
\end{coq}
\begin{coqout}{title=Response,width=6cm}
1 subgoal
  
  ============================
   3 = 3
\end{coqout}

Once in proof mode the system outputs the current proof state:
the list of named hypotheses on top of the horizontal bar (empty here)
and the current goal below the bar.

\begin{coq}{width=6cm,title=Ending a proof}
by reflexivity.
Qed.
\end{coq}
\begin{coqout}{title=Response,width=6cm}
No more subgoals.
my_first_lemma is defined.
\end{coqout}

\C{reflexivity} is a proof command that succeeds when the goal is an equality
and the two sides are equal up to computation.  In this case the two sides
are syntactically equal, examples involving computation will follow.

The \C{by} terminator is systematically used to prefix any proof sentence
that terminates a goal.  A proof sentence ends with a dot, and can
combine more than one proof command via the semi colon punctuation,
again more examples will follow.
The use of \C{by} is not limited to asserting no goals are left,
but is also used to close goals by trivial means.  The notion of triviality
is an important point we will discuss many times, for now lets just
say it is extensible by the user and that it includes reflexivity.
Hence the previous proof can be written as \C{by [].} where \C{[]}
denotes the empty list of proof commands.

\begin{coq}{title=Same proof}
Lemma my_first_lemma : 3 = 3.
Proof. by []. Qed.
\end{coq}

The \C{Qed} keyword terminates the proofs, checks it, and adds to
the library the new lemma.  We can hence access

Given that our first lemma is now part of the library as the result
of the \C{Qed} command we can access it.
\marginnote{should we use About ?}

\begin{coq}{width=6cm,title=Printing a lemma statement}
Check my_first_lemma.
\end{coq}
\begin{coqout}{title=Response,width=6cm}
my_first_lemma : 3 = 3
\end{coqout}

Before stating and proving a more interesting lemma let's look at a simple
but very important detail: its name.  The name is very badly chosen since
it is hard to guess what the lemma is about by just reading its name.
The name of a lemma is used whenever a proof appeals to such result,
similar to what a standard mathematical text would do by a reference or
a citation like ``this holds by lemma (7.3.4)''.  Lemmas are given a name,
and not just a number, because names are easier to remember and are more
stable (i.e. do not depend on the section where the corresponding lemma
is proved and don't require a renaming if a lemma is inserted before them).
The name can, and should, convey as much information as possible wile
striving to stay short and handy.  A more reasonable name for our
first lemma would be \C{eq33}: it would clearly remind us it talks about
an equality.

The \mcbMC{} library follows a naming convention that is documented in
the reference manual \cite{xxx}.  Here we introduce it by examples.

To get a better feeling on how lemmas are named let's see some other
lemmas already proved for us.

\begin{coq}{width=5cm,title=Lemmas of the library}
Check add0n.
Check addnC.
\end{coq}
\begin{coqout}{title=Response,width=7cm}
add0n : forall n : nat, 0 + n = n
addnC : forall n m : nat, n + m = m + n
\end{coqout}

First of all, note the quantification. So far \C{forall} was used to quantify
over types and describe polymorphic functions.  Now it ranges over terms of
type \C{nat} and describes generic lemmas, holding for any natural numbers
\C{n} and \C{m}.

Going back to the naming conventions, note that both names begin with ``add'',
that is the principal operation involved.  In particular is the head symbol
(the most external one) in the left hand side of the equation.  The addition
operation is the one on natural numbers, hence then ``n'' component in
the suffix.

For the first equation, the other symbol involved in the name is ``0'',
that indeed plays an important role in the lemma.  The relative position
of ``0'' and ``n'' is chosen to recall to the user if the neutral element
is added to the left or to the right.  For example the lemma named
\C{addn0} would state \C{forall n : nat, n + 0 = n}.

Capital letters in the suffix denote standard properties.
In particular ``C'' stands for commutativity.  Standard properties are
always stated with ``macros'' (higher order predicates) to impose
consistency in their stating.  Indeed the real statement of
\C{addnC} is \C{(commutative addn)}, where \C{addn} is the
name of the addition operation and \C{commutative} is the
following macro.

\begin{coq}{title=Macro for stating commutativity (suffix ``C'')}
Definition commutative op := forall x y, op x y = op y x.
\end{coq}

Macros, being standard definitions, are completely transparent for the
system.\footnote{not yet for \C{Search}, but here I'm describing an ideal
situation}  Consistency is not only aesthetically, but ease the use of
lemmas: as we will see later, the order of the quantification may make
a lemma easier to use under certain circumstances.

The last general remark about naming convention worth making so early is that
uninteresting and recurrent lemmas must have a short names easy to identify and
disregard when reading a proof and at the same time easy to remember and short
to type when writing a proof.\marginpar{this is a mantra somewhat}

Going to a second example, let's see how quantified lemmas can be stated:

\begin{coq}{width=7cm,title=Stating quantified lemmas}
Lemma add0nmC (a : nat) (b : nat) :
  0 + (a + b) = b + a.
Proof.
\end{coq}
\begin{coqout}{title=Response,width=5cm}
1 subgoal

 a : nat
 b : nat
 =========================
  0 + (a + b) = b + a
\end{coqout}

Note the \C{a} and \C{b} before the colon share the same type.
Depending on your user interface and \Coq{} version it may be displayed
in a more compact way, putting  \C{a} and \C{b} on the same line.
We will use this form when possible.

The following and equivalent form is also accepted.  Note that
the user has to spell out the type only once.

\begin{coq}{title=Types annotations}
Lemma add0nmC (a b : nat) : 0 + (a + b) = a + b.
\end{coq}

Finally, since \C{a} and \C{b} are used as arguments to
the addition of natural numbers, their type can be completely
omitted and the system will infer it for us.

\begin{coq}{title=Stating quantified lemmas}
Lemma add0nmC a b : 0 + (a + b) = b + a.
\end{coq}

Now that we are satisfied with the statement, we llok into proving it.
At the current stage we need to use the two already existing
lemmas \C{add0n} and \C{addnC} and the fact that the equality predicate
is substitutive: given an equation we can replace any instance of
its left hand side with the corresponding instance of the right hand side.

The command is called rewrite, and its argument its called rule.

\begin{coq}{title=Pedestrian proof}
rewrite add0n.
rewrite addnC.
by [].
Qed.
\end{coq}

\begin{coqout}{title=Response line 1,width=6cm}
1 subgoal

  a, b : nat
  ============================
   a + b = b + a
\end{coqout}
\begin{coqout}{title=Response line 2,width=6cm}
1 subgoal

  a, b : nat
  ============================
   b + a = b + a
\end{coqout}

Explain the job of instantiation rewrite does: picks \C{(a + b)} for
\C{n} in \C{add0n} and picks \C{a} for \C{n} and \C{b} for \C{m} in
\C{addnC}.  Explain it is guided by the head symbol \C{+} and traverses
the goal left to right.  Define "pattern".

\begin{coq}{width=4cm,title=The new lemma}
Check add0nmC.
\end{coq}
\begin{coqout}{title=Response,width=8cm}
add0nmC : forall a b : nat, 0 + (a + b) = b + a.
\end{coqout}

The rewrite proof command is very recurrent, hence a shorter form
using multiple rules.

\begin{coq}{title=More compact proof}
Lemma add0nmC a b : 0 + (a + b) = b + a.
Proof. by rewrite add0n addnC. Qed.
\end{coq}

Note that the equations are used in order, e.g. the script

\begin{coq}{title=Wrong proof}
Lemma add0nmC_bad a b : 0 + (a + b) = b + a.
Proof. by rewrite addnC add0n. Qed.
\end{coq}

Fails because \C{addnC} flips \C{0} with \C{(a + b)}, then
\C{add0n} finds no matching sub term.  In any case the proof
is wrong, commutativity has to be employed to flip \C{a} and \C{b},
or \C{b} and \C{a}, but not \C{0}.  If one wants to keep the rules
in that order he needs to drive \Coq{} by better specifying the
pattern he infers from \C{addnC}.

\begin{coq}{title=Rewrite pattern}
Lemma add0nmC a b : 0 + (a + b) = b + a.
Proof. by rewrite [a + _]addnC add0n. Qed.
\end{coq}

Explain pattern: overrides the pattern inferred.
Explain underscore: saves space if b is large.

Discuss: stating an equation (LHS is bigger).

\subsection{Equality up to computation}

As we said at the beginning, equality is up to computation.
Since \C{(0 + 3)} computes to \C{3}, proving the equality among
the two expressions is trivial.

\begin{coq}{title=Computation in proofs}
Eval compute in (0 + 3).  (* prints 3 *)
Eval compute in (true || false).  (* prints true *)
Lemma example3 : 0 + 3 = 3.
Proof. by []. Qed.
\end{coq}

Indeed \C{reflexivity}, and \C{by []} as well, take computation
into account.
Note it also works for symbolic expressions.

\begin{coq}{title=Computation in proofs}
Lemma add0n n : 0 + n = n.
Proof. by []. Qed.
\end{coq}

Explain addn0 is different, why, and that it will be shown later,
when induction will be introduced.

If we go back playing with the previous proof, we see that the
last rule can be omitted.

\begin{coq}{title=This also works}
Lemma add0nm_bad a b : 0 + (a + b) = b + a.
Proof. by rewrite [a + _]addnC. Qed.
\end{coq}

The reason is that \C{(0 + (b + a))} is equal
to \C{(b + a)} up to computation.
Note that add0n  is not only for free (in its proof) but
also in its use.

Maybe talk about \C{/=} here.

\begin{coq}{title=This also works}
Lemma add0nm_bad a b : 0 + (a + b) = b + a.
Proof. by rewrite /= addnC. Qed.
\end{coq}

Note the pattern can go, since \C{/=} leaves the goal with
no ambiguity.
Note that, if we mind computation, the lemma is just an alias
for \C{addnC} and hence is useless.  It may feel depressing, but
it is actually a very important hygiene measure: only useful lemmas
are kept, trade off between all combinations or few base blocks that
combine well (harder in principle, requires a language that glues well).

\subsection{Truth of a boolean predicate}

Special case of equations

\begin{coq}{title=Stating that a boolean predicate holds}
Lemma odd2n7 n : odd (2 * n + 7) = true.
\end{coq}

When we use it:

\begin{coq}{title=Stating that a boolean predicate holds}
Lemma example x : odd (2 * x + 7) || odd (2 * x) = true.
Proof. by rewrite odd2n7. Qed.
\end{coq}

Recall \C{(true || x)} reduces to \C{true}, hence the
goal becomes trivially \C{(true = true)} after the
rewriting step.

Also (coq user) remark that here we don't go left/right, or split
a conjunction, we can simply rewrite and let computation clean up/progress.

This is so frequent that \C{= true} is hidden to the user.
One can write

\begin{coq}{title=Stating a boolean lemma}
Lemma odd2n7 n : odd (2 * n + 7).
\end{coq}

This is not ambiguous: statements must have type Prop, while
the expression has type bool.  The system hence inserts
the \C{= true} automatically.  Note that it is also hidden,
so you have to learn that a lemma not explicitly showing the
= sign can be used by rewrite.

% \subsection{old}
% 
% \begin{itemize}
% \item Coq comes with an equality prexdicate, that we see as axiomatic for the
% moment (in this section). It is called \C{eq} and has an infix
% notation \C{=}. Example: \C{Check 3 = 3}, \C{Check 3 = 4}.
% \item Observe that, \C{Check} just checks the well-formedness of the
%   statement, not its truth. Well-formedness here means that \C{=}
%   requires both hand-sides to be of the same type.
%   Example \C{Check 3 = [::4]}. Observe that statement also have a
%   type, called \C{Prop}.
% \item We can give a name to a statement, that we would like to prove
%   formally. Example \C{Lemma foo : 3 = 3}. Observe what happens at the
%   level of the interface (we start a proof).
% \item And now, how to prove equality statements. Observationaly, \C{=}
%   behaves as expected: this is trivally true because \C{=} is
%   reflexive.
% \item End of the proof, \C{Qed}.
% \item Now we have a theorem called \C{foo} at hand in the
%   environement: \C{Check foo}. We can use it but it is useless...
% \item In fact the libraries you load provide theorems that you can use
%   in your proofs. Example:  \C{Check add0n}, \C{Check addnC}. Note the
%   universal quantification.
% \item How to state lemmas like this:
% \C{Example toto a b : 0 + (a + b) = a + b}.
% \item Equality is not just reflexive, it is also substitutive. This is
%   implemented via the \C{rewrite} tactic, that we can use to prove
%   this lemma, using \C{add0n}. Note that the tactic found the right
%   instance of \C{add0n} to be used, with \C{(a + b)}.
% \item \C{Qed}, \C{Check}, observe the universal quantification.
% \item Rewrite steps can be chained, in order. Example:
% \C{Example toto a b : 0 + (a + b) = b + a}. can be proved by rewriting
% \C{add0n} then, \C{addnC}. Note that \C{addnC} then \C{add0n} does
% something different.
% \item In fact reflexivity is more than syntactic equality. Prove
% \C{3 + 1 = 4}, \C{true || false = true}. But also with variables,
% e.g. \C{add0n}, connect with the programs written in the previous
% chapters.
% 
% \item More advanced forms of proofs by the \C{rewrite} tactic allow to
%   select a pattern and to simplify the goal for free by
%   computation. Hence \C{addn0} just does not need to be stated
%   explicitely in the proof script. Warning: simpl is a Pandora
%   box. This should come very early
%   and go with/before boolean reflection style proofs by rewriting.
% 
% \item Special case of equality statements that play a pervasive role
%   throughout the libraries: \C{_ = true}. For instance
%   \C{odd (2 * x + 7) = true}. This is both a 'logical' statement,
%   saying that any expression \C{(2x+7)} is an odd number, and
%   an equation that can be rewriten during the course of a proof. Find
%   an example, like \C{odd (2 * x + 7) || odd (2 * x)}. rewrite and
%   compute.
% 
% \item Now we should point the coercion, because people will search and
%   check.
% \end{itemize}



\mcbLEARN{case}
\mcbREQUIRE{}
\mcbPROVIDE{case, -, indentation, brute force proofs, semicolon, => after case}
\mcbLEVEL{1}
\mcbsection{Case analysis}

In programming inductive data is matched against to consider all
possible cases.  In the same way, when we reason on a program
whose input is symbolic, we can proceed by considering all possible
cases.

\begin{coq}{title=First De Morgan's law: pedestrian proof}
Lemma negb_and (a b : bool) : ~~ (a && b) = ~~ a || ~~ b.
Proof.
case: a.
- by []
case: b.
- by [].
by [].
Qed.
\end{coq}
\begin{coqout}{title=Response line 3,width=8cm}
2 subgoals
  
  b : bool
  ============================
   ~~ (true && b) = ~~ true || ~~ b

subgoal 2 is:
 ~~ (false && b) = ~~ false || ~~ b
\end{coqout}

Explain the decorator and the indentation policy.
Note a removed from context, substituted by true and false.
Note first goal computes to b=b.
Note second goal does not, computation blocked, we case b.

\begin{coqout}{title=Response line 5,width=8cm}
2 subgoals
  
  ============================
   ~~ (false && true) = ~~ false || ~~ true

subgoal 2 is:
 ~~ (false && false) = ~~ false || ~~ false
\end{coqout}

Now all computes.  We are not satisfied, we could write
as the only tactic \C{case: a; case: b.}.

\begin{coqout}{title=Response,width=8cm}
4 subgoals
  
  ============================
   ~~ (true && true) = ~~ true || ~~ true

subgoal 2 is:
 ~~ (true && false) = ~~ true || ~~ false
subgoal 3 is:
 ~~ (false && true) = ~~ false || ~~ true
subgoal 4 is:
 ~~ (false && false) = ~~ false || ~~ false
\end{coqout}

Explain semicolon.
Note that all computes. Note that this is always possible
for finitary data, it amounts to comprehensive testing.

\begin{coq}{title=First De Morgan's law}
Lemma negb_and (a b : bool) : ~~ (a && b) = ~~ a || ~~ b.
Proof. by case: a; case: b. Qed.
\end{coq}

other examples, exercises, can be

\begin{coq}{title=possible exampels}
Lemma andbA : associative andb.        Proof. by do 3!case. Qed.
Lemma orbCA : left_commutative orb.    Proof. by do 3!case. Qed.
Lemma andbN b : b && ~~ b = false.     Proof. by case: b. Qed.
\end{coq}

to explain A and N suffixes, plus do iteration, plus stack top.

When data is not finitary, then case is not sufficient.

\begin{coq}{}
Lemma eqnxx (x : nat) : eqn x x = true.
Proof. case: x => [|x'].
\end{coq}

Name subterms.

\begin{coqout}{title=Response,width=8cm}
  x' : nat
  ============================
   eqn x'.+1 x'.+1 = true
\end{coqout}

But before induction we look at implication

\mcbLEARN{apply}
\mcbREQUIRE{}
\mcbPROVIDE{apply, ->, =>, //}
\mcbLEVEL{1}
\mcbsection{Implication and backchaining}

\begin{coq}{width=4cm}
Check @dvdn_mul.
\end{coq}
\begin{coqout}{title=Response,width=8cm}
dvdn_mul : forall d1 d2 m1 m2 : nat,
  d1 %| m1 -> d2 %| m2 -> d1 * d2 %| m1 * m2
\end{coqout}

Explain implication symbol, the ambiguity with function space will
be explained later on.

\begin{coq}{}
Lemma stupid x : 3 * x %| 6 * x^2.
Proof.
apply: dvdn_mul.
- by [].
apply: dvdn_mulr.
exact: dvdnn.
Qed.
Lemma stupid2 x : 3 * x %| 6 * x^2.
Proof.
rewrite dvdn_mul // dvdn_mulr // dvdnn.
Qed.
\end{coq}

How do we prove this kind of lemmas

\begin{coq}{}
Lemma stupid3 x y : x %| y -> 3 * x %| 6 * y.
Proof.
move=> div_xy.
by apply: dvdn_mul.
Qed.
\end{coq}

by looks in the hyps.

\begin{coq}{}
Lemma stupid3 x y (div_xy : x %| y) : 3 * x %| 6 * y.
Proof. exact: dvdn_mul. Qed.
\end{coq}

equivalent syntax, uncommon.

Make another example working on the stack

\begin{coq}{}
Lemma andb_idl (a b : bool) : (b -> a) -> a && b = b.
Proof. by case: a; case: b => // ->. Qed.
\end{coq}

note // in =>, note -> in =>, note runs on 4 goals.

one last example with true=false in the context.

\section{Proofs by induction}

\begin{coq}{}
Lemma eqnxx x : eqn x x.
Proof.
elim: x=> [|x' IH].
rewrite /=.
apply: IH.
by [].
Qed.
Lemma eqnxx x : eqn x x.
Proof. by elim: x. Qed.
\end{coq}

requires induction.

\section{Curry-Howard}

Explain CH for arrows, foralls, equality and induction. Show Proof for
some examples of proofs.

First implication, start with comparing the source and target of the
arrows that are common to this chapter and the preceeding.

\begin{coq}{}
Inductive |*list*| (A : Type) : Type :=
    nil : list A | cons : A -> list A -> list A

list: Type -> Type

Definition |*is_equal_to_2*| (x : nat) : x = 2.

is_equal_to_2 : nat -> Prop
\end{coq}






\section{To be sorted}

\subsection{Universal quantification}

\begin{coq}{}
Variable A : Prop.
Definition toto : A -> A := fun x => x.
Lemma  toto : A -> A.
 Proof.
  move => x.
  Show Proof.
  apply: x.
 Qed.
\end{coq}

Note : \C{toto :  A -> A : Prop}
discuss interactive proof construction, other example with a real
apply that open subgoals.

In CH style we see a proof, where lambdas/apply work too (same tactics)

Specialization of an quantified lemma via application (maybe also
\C{move/(_ x) in H}).
Soft intro of the work of unification (FO) during application (infer
arguments of conclusion's predicate symbol \C{prime _}).

We should also display her  HO "predicators": commutative.  Discuss
order of quantification in transitive and similar, naming conventions

\subsection{Curry-Howard for equality proofs}


Show \C{rewrite -H}, plus the idea that one may want to select
occurrences (with simple patterns).


May be should we say rather early that rewriting under binders is not
allowed by the system.

\subsection{proof term for rewrite (CH)}

There are also proof terms for equality proofs (show \C{refl})
The work rewrite does is non trivial (infer P)

\begin{coq}{}
Check eq_rect (* : forall P : nat -> Prop, .... forall n, P n *)
\end{coq}

Write a proof term by hand.
(note that it works because of beta being part of conversion).
See that rewrite does the verbose part for you (infer P).

An example of ambiguous P from above, driving rewrite means driving the
synthesis of P.

\subsection{other eq related tactics}

Congr, injection \C{[= ]}, discriminate (//), \C{->}.


\subsection{proof term for induction}

Again show that it boils down to infer P


\subsection{proofs by induction in their generality}

Show that even for commutativity of addition one needs to
"load the goal" (or to help synthesize a more general P).


\section{Exercises (explained)}

xor-odd, then cancel encode decode, or something on primes.

Take the occasion to present last/first, bullets, by, syntaxes for
\C{=> [|IH x xs]} after a case.


\section{Notes}

% From calculability to proofs, hence the CC, and the fact that
% reasoning principles without a computational content become axioms.
%
% This is a non technical chapter and message should be:
% \begin{itemize}
% \item instantiation of a universal statement is application (also the pair)
% \item Excluded middle is not available by default (choice?)
% \item Conversion as a pervasive indistinguishably, what inside
%   (beta, definition unfolding,...)
% \item Dependent types: eq, sigma (which example?)
% \end{itemize}
%
% One options is: avoid relating type theory and other logics. We say:
% we have a formal game where the basic elements are programs/functions
% that come with types to avoid confusion. full stop. (no relation with
% proof theory, set theory). maybe mention that roots are in calculability (hence
% the choice to pick functions as primitive and not sets). This is lucky because
% (computable) functions are today executable by a computer.  Still not all
% concepts are "computable" hence some principles are problematic: EM,.... we
% mainly stay in the lucky fragment (again no propaganda on intuitionistic logic,
% constructive math; just a mention).
%

running topic: statements and their proofs. This chapter provides a gentle
introduction to the implicative fragment of the logic and to the
interactive construction of formal proofs. The reference to
Curry-Howard should come as late as possible. It should indeed be
possible to present a substantial amount of example and intuitions
before explaining this. Specially if we carefully use the automated
introduction of variables bound before the semi-column. At the tactic
level, we should be able to avoid the need for Curry Howard related
constructions like \C{apply: (H x)} in the first part of the chapter.

\begin{itemize}
\item \C{Prop} as the type of statements
\item forall-lambda, predicates as functions to Prop.  The difference with the forall in the previous chapter is that the type the var ranges in is not "Type" but a data type like nat.
\item examples with implicative predicate logic, modus ponens
\item definition = lemma (proofs of implications are lambdas-app)
\item tactics to generate the terms in a less pedestrian way (move, apply)
\item if one calls Show Proof in the middle he sees one is building
	the proof term incrementally, as one draws a proof tree (in CH style)
\end{itemize}

Now we make the point of Qed, the kernel checks the term produced via
tactics (or hand-written).

Examples of predicates: equality
\begin{itemize}
\item not to talk about indexes of inductive families we introduce eq as
	an axiom and refl as another axiom to prove eq and we insist on
	conversion
\item examples, among which a beta expansion (to help later on with elimination)
\item now, what are the proofs of an equality? maybe we start with rewrite
\item examples are symmetry, transitivity
\item then we give the elimination principle as an axiom, and explain the
	work rewrite does in synthesizing the predicate
\item the job of rewrite is not trivial nor un-ambiguous: to drive the synthesis show occurrence selection (later on we see the same problem for elim)
\item other exercise on other eq related "commands" like discriminate, injection arrows, congr.
\end{itemize}

Showing universal properties on inductive data
\begin{itemize}
\item first enouce some lemmas on concrete examples, like not true = false,
	then try to say "froall b : bool, ...".
\item first on enumerated types (bool) via case
\item them on nat via case and elim
\item then show the proof term as an application of nat-rect, again the problem
	is to write down P and elim does that.
\item here an additional problem: loading the goal before using it to generate P
\end{itemize}

We should manage to prove stuff on the concepts defined in chapter 1,
notably \C{odd(a+b) = xor (odd a) (odd b)}.  But before you need lemmas
on "odd (S x) = not (odd x)", to make the point that the proof is trivial with
a library that has all the needed facts.  Also the base case goes away by ssr (in general spell out where the computation saves work).

Comparison with other approaches:
\begin{itemize}
\item Compare an axiomatic, equational presentation of arithmetic to
  its formalization as an inductive type with functions that
  compute.
\end{itemize}


Another example is set-nth of rcons as done by Florent Hivert, that has not
developed the theory of set-nth with cat, and hence messes up the proof
that gets shorter if one does the homework.

Another example could be a proof that requires the induction principle
on nat that (strong/course of values induction).

A maybe good example that forces you to do patterns or occ numbers and explain
that 2  contains syntactically 0 is the proof that code/decode cancel (in
choice).

We lack example with a mathematical interest, maybe arithmetic is sufficient like binomial.v.

We need also /= and hence Arguments simple never (and we try to omit nosimpl).
Maybe we should document that nosimpl has been superseeded by
the option of Arguments in the manual.
Let say that controlling reduction is an important topic when you
do ssr style.

In this chapter we present several features of the proof language, but it it is
not about the proof language itself (reference manual). What one adds to the
reference manual here is an example of usage in the right context of the
commands.  E.g. we give an hint of what patterns do (rewrite) but we don't
discuss all the matching discipline... nor the most advanced syntaxes)
