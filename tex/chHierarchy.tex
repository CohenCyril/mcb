\Chapter{Hierarchies}{ Organizing knowledge}

Packaging records, the bigop hierarchy.
Scaling with packed classes and mixins, to the ssralg
hierarchy. Presentation of the content of ssralg in terms of structures
and of the theory? Should the latter be a separate chapter.

Maybe a plugin for a new vernacular to script the creation/declaration
of structures/instances so that the level basic can touch the argument
easily.

Explain what the abstraction barrier is (like unfolding a GRing projection)

\gotcha{if you see GRing.toto then you broke an abstraction barrier}

I guess here one explains the interfaces and gives examples of the instances.
It is unclear how one can explain the hierarchy without giving examples,
and examples are usually sub-types (ord, Zp, perm).  So maybe ch6 should
come before ch5.

%%%%%%%%%%%%%%%%%%%%%%%%%%%%%%%%%%%%%%%%%%%%%%%%%%%%%%%%%%%%%%%%%%%
\mcbLEVEL{1}
\mcbLEARN{contents}
\mcbPROVIDE{}
\mcbREQUIRE{}
\mcbsection{Guided tour of widespread eqtype}


Examples: nat (but not ordinals yet, since we need ch 6) ordinals, ...

