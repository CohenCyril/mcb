\Chapter{Hierarchies}{Organizing knowledge}

Packaging records, the bigop hierarchy.
Scaling with packed classes and mixins, to the ssralg
hierarchy. Presentation of the content of ssralg in terms of structures
and of the theory? Should the latter be a separate chapter.

Maybe a plugin for a new vernacular to script the creation/declaration
of structures/instances so that the level basic can touch the argument
easily.

Explain what the abstraction barrier is (like unfolding a GRing projection)

\gotcha{if you see GRing.toto then you broke an abstraction barrier}

I guess here one explains the interfaces and gives examples of the instances.
It is unclear how one can explain the hierarchy without giving examples,
and examples are usually sub-types (ord, Zp, perm).  So maybe ch6 should
come before ch5.

%%%%%%%%%%%%%%%%%%%%%%%%%%%%%%%%%%%%%%%%%%%%%%%%%%%%%%%%%%%%%%%%%%%
\mcbLEVEL{1}
\mcbLEARN{Packed class organisation of types}
\mcbPROVIDE{packed classes, GRing}
\mcbREQUIRE{simplified eqType, finType}
\mcbsection{Structure hierarchies}
%%%%%%%%%%%%%%%%%%%%%%%%%%%%%%%%%%%%%%%%%%%%%%%%%%%%%%%%%%%%%%%%%%%


\subsection{Telescopes}

use monoid hierarchy (mention subType hierarchy)\\
point out nested projections\\
explain gradual instanciation\\

\subsection{Packed classes}

start from eqType\\
show at least choiceType \& ringType\\
dual coercion/instance inheritance graph\\
dependent mixins\\
cloning, constructors (with forward ref)

\subsection{Module interface}

internals/implementation/exports/theory\\
type parameters\\
exports list\\
``protected'' internals

\subsection{Constructors}

the interface\\
phantom\_id\\
transport\\
generic cloning

%%%%%%%%%%%%%%%%%%%%%%%%%%%%%%%%%%%%%%%%%%%%%%%%%%%%%%%%%%%%%%%%%%%
\mcbLEVEL{1}
\mcbLEARN{the collection hierarchy; inheritance by coercion}
\mcbPROVIDE{\\in}
\mcbREQUIRE{eqType, seq, coercions}
\mcbsection{Predicates and sets}
%%%%%%%%%%%%%%%%%%%%%%%%%%%%%%%%%%%%%%%%%%%%%%%%%%%%%%%%%%%%%%%%%%%

\subsection{Collective predicates}
in\\
PredType\\
applicative\_pred and collective\_pred classes\\
simplPred\\
mem\\
bounded quatification

\subsection{Coercion bierarchies}

PredArgType\\
pred\_class\\
PredArgType\\
\{set \_\}, \{group \_\}

\subsection{Qualifiers}
\subsection{Generic predicates}

%%%%%%%%%%%%%%%%%%%%%%%%%%%%%%%%%%%%%%%%%%%%%%%%%%%%%%%%%%%%%%%%%%%
\mcbLEVEL{1}
\mcbLEARN{canonical properties of operators and predicates; keyed predicates}
\mcbPROVIDE{linear, semiring\_closed}
\mcbREQUIRE{bigop}
\mcbsection{Lemma overloading}
%%%%%%%%%%%%%%%%%%%%%%%%%%%%%%%%%%%%%%%%%%%%%%%%%%%%%%%%%%%%%%%%%%%

What is lemma overloading

\subsection{Monoid properties}

bigop lemmas and their use

\subsection{Linearity properties}

morphism properties\\
the chaining issue.

\subsection{General linearity}

helper operator classes

\subsection{Algebraic predicates}

closure properties\\
bounded quantification

\subsection{Keyed predicates}

keyed predicates\\
non-nesting telescopes\\
default keying

%%%%%%%%%%%%%%%%%%%%%%%%%%%%%%%%%%%%%%%%%%%%%%%%%%%%%%%%%%%%%%%%%%%
\mcbLEVEL{3}
\mcbLEARN{reflected notation}
\mcbPROVIDE{mxdirect}
\mcbREQUIRE{bigop}
\mcbsection{Quotation}
%%%%%%%%%%%%%%%%%%%%%%%%%%%%%%%%%%%%%%%%%%%%%%%%%%%%%%%%%%%%%%%%%%%

Advanced topic\\
contrast with CA (maple, etc)

\subsection{Quotation by notation}

static overloading\\
explicit interpreter\\
presentations\\
algebraic terms

\subsection{Quotation by inference}

direct sums\\
default instances\\
definition unfolding\\
double keying


