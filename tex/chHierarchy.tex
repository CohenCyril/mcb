\Chapter{Organizing Theories}{}
\label{ch:hierarchy}

% Packaging records, the bigop hierarchy.
% Scaling with packed classes and mixins, to the ssralg
% hierarchy. Presentation of the content of ssralg in terms of structures
% and of the theory? Should the latter be a separate chapter.

% Maybe a plugin for a new vernacular to script the creation/declaration
% of structures/instances so that the level basic can touch the argument
% easily.

% Explain what the abstraction barrier is (like unfolding a GRing projection)

% \gotcha{if you see GRing.toto then you broke an abstraction barrier}

% I guess here one explains the interfaces and gives examples of the instances.
% It is unclear how one can explain the hierarchy without giving examples,
% and examples are usually sub-types (ord, Zp, perm).  So maybe ch6 should
% come before ch5.

We have seen in the last two chapters how inferred dependent records
--- \emph{structures} --- are an efficient means of endowing
mathematical objects with their expected operations and properties.
So far we have only seen single-purpose structures: \C{eqType}
provides decidable equality, \C{subType} an embedding into a
representation type, etc.

However, the more interesting mathematical objects have \emph{many}
operations and properties, most of which they share with other kinds
of objects: for example, elements of a field have all the properties
of those of a ring (and more), which themselves have all the
properties of an additive group. By organizing the corresponding
\mcbCIC{} structures in a hierarchy, we can materialize these
inclusions in the \mcbMC{} library, and share operations and properties
between related structures. For example, we can use the same generic
ring multiplication for rings, integral domains, fields, algebras, and
so on.

Organizing structures in a hierarchy does not require any new logical
feature beyond those we have already seen: type inference with
dependent types, coercions and canonical instances of structures.  It
is only a ``simple matter of programming'', albeit one that involves
some new formalisation idioms. This chapter describes the most
important: telescopes, packed classes, and phantom parameters.
% class coercions, and quotation.

While some of these formalisation patterns are quite technical, casual
users do not need to master them all: indeed the documented interface
of structures suffices to use and declare instances of structures. We
describe these interfaces first, so only those who wish to extend old
or create new hierarchies need to read on.

%%%%%%%%%%%%%%%%%%%%%%%%%%%%%%%%%%%%%%%%%%%%%%%%%%%%%%%%%%%%%%%%%%%
\mcbLEVEL{1}
\mcbLEARN{Using a structure}
\mcbPROVIDE{\lib{fintype}, \C{GRing}, \lib{zmodp}}
\mcbREQUIRE{\C{eqType}}
\mcbsection{Structure interface}\label{str:itf}
%%%%%%%%%%%%%%%%%%%%%%%%%%%%%%%%%%%%%%%%%%%%%%%%%%%%%%%%%%%%%%%%%%%

Most of the documented interface to a structure concerns the operations and
properties it provides. This will be obvious from the embedded
documentation of the \lib{ssralg} library, which provides structures for
most of basic algebra (including rings, modules, fields). While these
are of course important, they pertain to elements of the structure
rather than the structure itself, and indeed are usually defined
outside of the module introducing the structure.

The intrinsic interface of a structure is much smaller, and consists mostly
of functions for creating instances to be typically
declared \C{Canonical}. For structures like \C{eqType}
that are packaged in a submodule (\C{Equality} for \C{eqType}), the
interface coincides with the contents of the \C{Exports} submodule.
For \C{eqType} the interface comprises:
\begin{itemize}
\item \C{eqType}: a short name for the structure type (here,
  \C{Equality.type})
\item \C{EqMixin}, \C{PcanEqMixin}, \lstinline/[eqMixin of T by <:]/:
  mixin constructors that bundle the \emph{new} operations and
  properties the structure provides.
\item \C{EqType}: an instance constructor that creates an instance of
  the structure from a mixin.
\item \lstinline/[eqType of T]/, \lstinline/[eqType of T for S]/:
  cloning constructors that specialize a canonical or given instance
  of the structure (to \C{T} here).
\item canonical instances and coercions that link the structures to lower ones
  in the hierarchy or to its elements, e.g., \C{Equality.sort}.
\end{itemize}
\index[coq]{\C{eqType}}
\index[coq]{\C{Equality.type}}
\index[coq]{\C{Equality.sort}}
\index[coq]{\C{EqMixin}}
\index[coq]{\C{PcanEqMixin}}
\index[coq]{\C{[eqMixin of .. by <:]}}
\index[coq]{\C{EqType}}
\index[coq]{\C{[eqType of ..]}}
Canonical instances and coercions are not mentioned directly in the
documentation because they are only used indirectly, through type
inference; a casual user of a structure only needs to be aware of which other
structure it extends, in the hierarchy.

Let us see how the creation operations are used in practice, drawing
examples from the \lib{zmodp} library that puts a ``mod $p$'' algebraic
structure on the type \C{ordinal p} of integers less than $p$.
The \C{ordinal} type is defined in library \lib{fintype} as follows:

\begin{coq}{name=zp0}{}
Inductive ordinal n := Ordinal m of m < n.
Notation "'I_ n'" := (Ordinal n).
Coercion nat_of_ord n (i : 'I_n) := let: @Ordinal _ m _ := i in m.
\end{coq}

Algebra only makes sense on non-empty types, so \lib{zmodp} only defines
arithmetic on \lstinline/'I_p/ when \C{p} is an explicit successor.
This makes it easy to define \C{inZp}, a ``mod $p$'' right inverse to
the \lstinline/ordinal >-> nat/ coercion, and a 0 value. With these
the definition of arithmetic operations and the proof of the basic
algebraic identities is straightforward.

\begin{coq}{name=zp1}{}
Variable p' : nat.
Local Notation p := p'.+1.
Implicit Types x y : 'I_p.
Definition inZp i := Ordinal (ltn_pmod i (ltn0Sn p')).
\end{coq}
The \C{inZp} construction injects any natural number \C{i}
into \C{'I_p} by applying the modulus.  Indeed the
type of \C{ltn_pmod} is \C{(forall m d : nat,0 < d -> m \%\% d < d)},
and \C{(ltn0Sn p')} is a proof that \C{(0 < p)}.

We can now build the $\mathbb{Z}$-module operations and
properties:

\begin{coq}{}{}
Definition Zp0 : 'I_p := ord0.
Definition Zp1 := inZp 1.
Definition Zp_opp x := inZp (p - x).
Definition Zp_add x y := inZp (x + y).
Definition Zp_mul x y := inZp (x * y).

Lemma Zp_add0z : left_id Zp0 Zp_add.
Lemma Zp_mulC : commutative Zp_mul.
...
\end{coq}

Creating an instance of the lowest \lib{ssralg} structure, the
$\mathbb{Z}$-module (i.e., additive group), requires two lines:

\begin{coq}{name=zp4}{}
Definition Zp_zmodMixin := ZmodMixin Zp_addA Zp_addC Zp_add0z Zp_addNz.
Canonical Zp_zmodType := ZmodType 'I_p Zp_zmodMixin.
\end{coq}

Line 1 bundles the additive operations (0, $+$, $-$) and their
properties in a \emph{mixin}, which is then used in line 2 to create a
canonical instance. After line 2 all the additive algebra provided in
\lib{ssralg} becomes applicable to \lstinline/'I_p/; for example \C{0}
denotes the zero element, and \C{i + 1} denotes the successor of \C{i} mod $p$.

The \C{ZmodMixin} constructor infers the operations \C{Zp_add}, etc.,
from the identities \C{Zp_add0z}, etc.. Providing an explicit definition
for the mixin, rather than inlining it in line 2, is important as it
speeds up type checking, which never need to open the mixin bundle.

The first argument to the instance constructor \C{ZmodType} is
somewhat redundant, but documents precisely the type for which this
instance will be used.
This can be important as the value
inferred by \Coq{} could be ``different''.
Indeed line 2 does perform some nontrivial inference, because,
although \C{zmodType} is the first \lib{ssralg} structure, it is not at
the bottom of the hierarchy: in particular \C{zmodType} derives from
\C{eqType}, so the \C{==} test can be used on all \C{zmodType}
elements.  The \C{ZmodType} constructor infers a parent structure
instance from its first argument, then combines it with the mixin to
create a full \C{zmodType} instance.  This inference can have
the side effect of unfolding constants occurring in the description of the type
(though not in this case), that is the value on which the canonical solution is
indexed.  For this reason the type description, \C{'I_p} here, has to be
provided explicitly.
Section \ref{sec:phant} gives the technical details of instance
constructors.

Rings are the next step in the algebraic hierarchy. In order to
simplify the formalization of the theory of polynomials, %see section~\ref{sec:poly},
\lib{ssralg} only provides structures
for nontrivial rings, so we now need to restrict to \C{p} of the form
\lstinline/p'.+2/:

\begin{coq}{name=zp3}{}
Variable p' : nat.
Local Notation p := p'.+2.
Lemma Zp_nontrivial : Zp1 != 0 :> 'I_p. Proof. by []. Qed.
Definition Zp_ringMixin :=
  ComRingMixin (@Zp_mulA _) (@Zp_mulC _) (@Zp_mul1z _) (@Zp_mul_addl _) Zp_nontrivial.
Canonical Zp_ringType := RingType 'I_p Zp_ringMixin.
Canonical Zp_comRingType := ComRingType 'I_p (@Zp_mulC _).
\end{coq}

Line 6 endows \lstinline/'I_p/ with a \C{ringType} structure, making
it possible to multiply in \lstinline/'I_p/ or have polynomials over
\lstinline/'I_p/. Line 7 adds a \C{comRingType} commutative ring
structure, which makes it possible to reorder products, or distribute
evaluation over products of polynomials. Note that no mixin definition
is needed for line 8 as only a single property is added.

Constraining the shape of the modulus $p$ is a simple and robust way
to enforce $p>1$: it standardizes the proofs of $p>0$ and $p>1$, which
avoid the unpleasantness of multiple interpretations of \C{0}
stemming from different proofs of $p>0$ --- the latter tends to happen
with ad hoc inference of such proofs using canonical structures or
tactics. The shape constraint can however be inconvenient when the
modulus is an abstract constant (say, \lstinline/Variable p/), and \lib{zmodp}
provides some syntax to handle that case:

\begin{coq}{name=zp4}{}
Definition Zp_trunc p := p.-2.
Notation "''Z_' p" := 'I_(Zp_trunc p).+2.
\end{coq}

Although it is provably equal to \lstinline/'I_p/ when $p>1$,
\lstinline/'Z_p/ is the preferred way of referring to that type when
using its ring structure. Note that the two types are identical when
\C{p} is a \C{nat} literal such as 3 or 5.

Cloning constructors are mainly used to quickly create instances for
defined types, such as

\begin{coq}{name=zp5}{}
Definition Zmn := ('Z_m * 'Z_n)%type.
\end{coq}

While \lib{ssralg} and \lib{zmodp} provide the instances type inference
needs to synthesize a ring structure for \C{Zmn}, \Coq{} has to expand the
definition of \C{Zmn} to do so. Declaring \C{Zmn}-specific instances
will avoid such spurious expansions, and is easy thanks to cloning
constructors:

\begin{coq}{name=zp6}{}
Canonical Zmn_eqType := [eqType of Zmn].
Canonical Zmn_zmodType := [zmodType of Zmn].
Canonical Zmn_ringType := [ringType of Zmn].
\end{coq}

Cloning constructors are also useful to create on-the-fly instances
that must be passed explicitly, e.g., when specializing lemmas:

\begin{coq}{name=zp7}{}
have Zp_mulrAC := @mulrAC [ringType of 'Z_p].
\end{coq}

Finally, instances of \emph{join} structures that are just the union
of two smaller ones are always created with cloning constructors.  For
example, \lstinline/'I_p/ is also a finite (explicitly enumerable)
type, and the \lib{fintype} library declares a corresponding \C{finType}
structure instance. This means that \lstinline/'I_p/ should also have
an instance of the \C{finRingType} join structure (for \C{p} of the
right shape). This is not automatic, but thanks to the cloning
constructor requires only one line in \lib{zmodp}.

\begin{coq}{name=zp8}{}
Canonical Zp_finRingType := [finRingType of 'I_p].
\end{coq}


%%%%%%%%%%%%%%%%%%%%%%%%%%%%%%%%%%%%%%%%%%%%%%%%%%%%%%%%%%%%%%%%%%%
\mcbLEVEL{1}
\mcbLEARN{Simple telescope hierarchy}
\mcbPROVIDE{\lstinline/Monoid/ hierarchy}
\mcbREQUIRE{\lstinline/bigop/}
\mcbsection{Telescopes}
%%%%%%%%%%%%%%%%%%%%%%%%%%%%%%%%%%%%%%%%%%%%%%%%%%%%%%%%%%%%%%%%%%%

While using and populating a structure hierarchy is fairly
straightforward, creating a robust and efficient hierarchy can be more
difficult. In this section we explain \emph{telescopes}, one of the
simpler ways of implementing a structure hierarchy.  Telescopes
suffice for most simple --- tree-like and shallow --- hierarchies, so
new users do not necessarily need expertise with the more
sophisticated \emph{packed class} organization covered in the next
section.

Because of their limitations (covered at the end of this section),
telescopes were not suitable for the main type structure hierarchy
of the \mcbMC{} library, including \lib{eqtype}, \lib{choice}, \lib{fintype}
and \lib{ssralg}. However, as we have seen in
section~\ref{sec:bigoplemmas}, structures can be used to associate
properties to any logical object, not just types, and the
\C{Monoid.law} structure introduced in section~\ref{sec:bigoplemmas} is part
of a telescope hierarchy. Recall that \C{Monoid.law} associates an
identity element and Monoid axioms to a binary operator:

\begin{coq}{name=telescope10}{}
Module Monoid.
Variables (T : Type) (idm : T).
Structure law := Law {
  operator : T -> T -> T;
  _ : associative operator;
  _ : left_id idm operator;
  _ : right_id idm operator
}.
Coercion operator : law >-> Funclass.
\end{coq}
\index[coq]{\C{Law}}

The \C{Coercion} declaration facilitates writing generic \C{bigop}
simplification rules such as

\begin{coq}{name=telescope11}{}
  big1_eq R (idx : R) (op : Monoid.law idx) I r (P : pred I) :
    \big[op/idx]_(i <- r | P i) idx = idx
\end{coq}

Because the \C{Monoid} hierarchy is small, there is no need to bundle
the \C{Monoid.law} properties in a mixin. It thus takes only one line
to declare an instance for the boolean ``and'' operator \C{andb}

\begin{coq}{name=telescope12}{}
Canonical andb_monoid := Law andbA andTb andbT.
\end{coq}

This declaration makes it it possible to use \lstinline/big1_eq/ to
simplify the ``big and'' expression

\begin{coq}{name=telescope13}{}
\big[and/true]_(i in A) true
\end{coq}

to just \C{true},\footnote{There is no spurious
  \lstinline/add_monoid/ because the identity element is a manifest
  field stored in the structure type.}
as it informs type inference how to solve the unification problem
\lstinline/(operator $\mcbimplm{L}$)/ versus \C{andb}, by setting
\lstinline/$\mcbimplm{L}$/ to \C{andb_monoid}.

Now many of the more interesting \C{bigop} properties permute the
operands of the iterated operator; for example

\begin{coq}{name=telescope3}{}
pair_bigA: \big[op/idx]_i \big[op/idx]_j F i j = \big[op/idx]_p F p.1 p.2
\end{coq}

Such properties only hold for commutative monoids, so, in order to
state \lstinline/pair_bigA/ as above, we need a structure encapsulating
commutative monoids --- and one that builds on \C{Monoid.law} at that,
to avoid mindless duplication of theories. The most na\"{\i}ve way of
doing this, merely combining a \C{law} with a commutativity axiom,
works remarkably well. After

\begin{coq}{name=telescope20}{}
Structure com_law := ComLaw {
   com_operator : law;
   _ : commutative com_operator
}.
Coercion com_operator : com_law >-> law.
\end{coq}
\index[coq]{\C{ComLaw}}
At this stage, one can declare the canonical instance:

\begin{coq}{name=telescope21}{}
Canonical andb_comoid := ComLaw andbC.
\end{coq}
and then use \lstinline/big_pairA/ to factor nested ``big
ands'' such as

\begin{coq}{name=telescope22}{}
\big[andb/true]_i \big[andb/true]_j M i j.
\end{coq}

However, things are not so simple on closer examination: the
idiom only works because of the invisible coercions inserted during
type inference.

The definition of \lstinline/andb_comoid/ infers the implicit
\lstinline/com_operator/ argument \C{$\mcbimplm{L}$} of \C{ComLaw} by
unifying the expected statement
\lstinline/commutative (operator $\mcbimplm{L}$)/
of the commutativity property, with the actual
statement of \lstinline/andbC : commutative andb/. This finds
\lstinline/$\mcbimplm{L}$/ to be \lstinline/andb_monoid/ as above.

More importantly, the \C{op} in the statement of \lstinline/big_pairA/
really stands for \lstinline/operator (com_operator op)/. Thus
applying \lstinline/big_pairA/ to the term above leads to the
unification of \lstinline/operator (com_operator $\mcbimplm{C}$)/ with
\C{andb}. This first resolves the outer \C{operator}, as above, reducing the
problem to unifying \lstinline/com_operator $\mcbimplm{C}$/ with
\lstinline/andb_monoid/, which is finally solved using
\lstinline/andb_comoid/.  Hence, \lstinline/andb_comoid/ associates
commutativity with the \C{law} \lstinline/andb_monoid/ rather than the
operator \C{andb}, and the invisible chain of coercions guides the
instance resolution.

The telescope idiom works recursively for arbitrarily deep
hierarchies, though the \C{Monoid} one only has one more level for a
distributivity property

\begin{coq}{name=telescope30}{}
Structure add_law (mul : T -> T -> T) := AddLaw {
  add_operator : com_law;
  _ : left_distributive mul add_operator;
  _ : right_distributive mul add_operator
}.
Coercion add_operator : add_law >-> com_law.
\end{coq}
\index[coq]{\C{AddLaw}}

The instance declaration

\begin{coq}{name=telescope31}{}
Canonical andb_addoid := AddLaw orb_andl orb_andr.
\end{coq}
then associates distributivity to the \lstinline/andb_comoid/
structure which is inferred by the two-stage resolution process above,
and applying the iterated distributivity

\begin{coq}{name=telescope32}{}
bigA_distr_bigA :
  \big[times/one]_(i : I) \big[plus/zero]_(j : J) F i j
    = \big[plus/zero]_(f : {ffun I -> J}) \big[times/one]_i F i (f i).
\end{coq}
to \lstinline!\big[andb/true]_i /big[orb/false]_j M i j! involves
three-stage resolution.

The coercion chains that support the ease of use of telescope
hierarchies have unfortunately two major drawbacks: they limit the
shape of the hierarchy to a tree (with linear ancestry) and
trigger crippling inefficiencies in the type inference and type
checking heuristics for deep hierarchies.

%%%%%%%%%%%%%%%%%%%%%%%%%%%%%%%%%%%%%%%%%%%%%%%%%%%%%%%%%%%%%%%%%%%
\mcbLEVEL{2}
\mcbLEARN{Packed class organization of structures}
\mcbPROVIDE{packed classes, \C{GRing}}
\mcbREQUIRE{\C{eqType}}
\mcbsection{Packed classes}\label{sec:packed}
%%%%%%%%%%%%%%%%%%%%%%%%%%%%%%%%%%%%%%%%%%%%%%%%%%%%%%%%%%%%%%%%%%%

The type structure hierarchy for the \mcbMC{} library is both deep
(up to 11 levels) and non-linear. It is not uncommon for an algebraic
structure to combine the properties of two unrelated structures: for
example an algebra is both a ring and a module, neither of which is an
instance of the other. Thus the \mcbMC{} type structures are organized
along a different pattern, \emph{packed classes}, which is more flexible
and efficient than the telescope pattern, but requires more work to follow.

The packed class design calls for three layers of records for each
structure: a \emph{mixin} record holding the new operations and
properties the structure adds to the structures it extends (as in
section~\ref{str:itf}), a \emph{class} record holding all the
primitive operations and properties in the structure,
including those in substructures,
and finally a \emph{packed class} record that associates the class to
a type, and is used to define instances of the structure. Crucially,
in this organization, the type ``key'' that directs inference is always
a direct field of a structure's instance record, so all coercion
chains have length one.

This arrangement was already hinted at in section~\ref{rec:itf} while
commenting on the formalisation of the \C{eqType} structure,
which we recall here:

\begin{coq}{name=packed10}{}
Module Equality.

Definition axiom T (e : rel T) := forall x y, reflect (x = y) (e x y).

Record mixin_of T := Mixin {op : rel T; _ : axiom op}.
Notation class_of := mixin_of.

Structure type := Pack {sort :> Type; class : class_of sort}.
...
Module Exports.
Coercion sort : type >-> SortClass.
Notation eqType := type.
...
End Equality.
Export Equality.Exports.
\end{coq}
\index[coq]{\C{eqType}}

The \C{Exports} submodule, which we had omitted in section~\ref{rec:itf},
regroups all the declarations in \C{Equality} that should have global scope,
such as the \lstinline/Coercion/ declaration for \C{Equality.sort}.

The roles of mixin and class are conflated for \C{eqType}
because it sits at the bottom of the type structure hierarchy. To
clarify the picture, we need to move one level up, to the
\C{choiceType} structure that provides effective choice for decidable
predicates:

\begin{coq}{name=packed20}{}
Module Choice.

Record mixin_of T := Mixin {
  find : pred T -> nat -> option T;
  _ : forall P n x, find P n = Some x -> P x;
  _ : forall P : pred T, (exists x, P x) -> exists n, find P n;
  _ : forall P Q : pred T, P =1 Q -> find P =1 find Q
}.

Record class_of T :=
  Class {base : Equality.class_of T; mixin : mixin_of T}.

Structure type := Pack {sort; _ : class_of sort}.
\end{coq}
\index[coq]{\C{choiceType}}

% The class component of \C{type} is anonymous as this avoids
% spurious ``Projection value'' warnings; this is also done in the actual
% \mcbMC{} \C{Equality} module.\footnote{The code here and in
%   section~\ref{rec:itf} are slight simplifications of the real \mcbMC{} code.
%   We've also removed patches addressing historical inefficiencies on
%   the \Coq{} kernel heuristics, in particular a redundant copy of the
%   \C{sort} field appended to the \C{type} structure.}

The main operation provided by the \C{choiceType} structure \C{T} is a
choice function for decidable predicates
\C{(choose : pred T -> T -> T)} satisfying:


\begin{coq}{name=packed21}{}
Lemma chooseP P x0 : P x0 -> P (choose P x0).
Lemma choose_id P x0 y0 : P x0 -> P y0 -> choose P x0 = choose P y0.
Lemma eq_choose P Q : P =1 Q -> choose P =1 choose Q.
\end{coq}

%\marginnote{Say more about choice in exercises, like W lemmas.}
The mixin actually specifies a specific, depth-based,
strategy for searching and electing a witness: \C{choose} can be
defined using \C{find} and the axiom of countable
choice, which is derivable in \mcbCIC{}% \footnote{behind the
% \C{\{m|..\}} notation lies the \C{sig} dependent pair.  Such an inductive
% data type is a copy of the existential quantifier but is
% declared as a data type rather than a
% proposition~\cite[section 3.1.4]{Coq:manual}.
% \Coq{} checks that the \C{m} is built without using
% the existential witness \C{n}.  In other words the proof in
% input is only used to justify the termination of the
% search for \C{m}.}
.

\begin{coq}{}{}
Lemma find_ex_minn (P : pred nat)  :
  (exists n, P n) -> {m | P m & forall n, P n -> n >= m}.
\end{coq}
The stronger requirement
makes it possible to compose \C{choiceType}s, so that pairs or
sequences of \C{choiceType}s are also \C{choiceType}s.  This subtlety
is detailed in the comments of the \lib{choice} file.

An important difference to telescopes is that the definition of
\C{Choice.type} does not link it directly to \C{eqType}: a
\C{choiceType} structure contains an \C{Equality.class_of} record, rather
than an \C{eqType} structure. That link needs to be constructed
explicitly in the code that follows the definition of \C{Choice.type}:

\begin{coq}{name=packed22}{}
Coercion base : class_of >-> Equality.class_of.
Coercion mixin : class_of >-> mixin_of.
Coercion sort : type >-> Sortclass.
Variables (T : Type) (cT : type).
Definition class := let: @Pack _ c as cT' := cT return class_of cT' in c.
Definition eqType := Equality.Pack class.
\end{coq}

Here \C{class} is just the explicit definition of the second component
of the \lstinline/Section/ variable \C{cT : type}.

Thanks to the \lstinline/Coercion/ declarations, the \C{eqType} definition is
indeed the \C{eqType} structure associated to \C{cT}, with \C{sort}
equal to \C{cT}${}\equiv{}$\C{sort cT} and \C{class} equal to \C{base class}.
The actual link between \C{choiceType} and \C{eqType} is established
by the following two lines in \C{Choice.Exports}:

\begin{coq}{name=packed23}{}
Coercion eqType : type >-> Equality.type.
Canonical eqType.
\end{coq}

Line 1 merely lets us explicitly use a \C{choiceType} where an
\C{eqType} is expected, which is rare as structures are almost alway
implicit and inferred.  It is line 2 that really lets \C{choiceType}
extend \C{eqType}, because it makes it possible to use any
\emph{element} \C{(T : choiceType)} as an element of an \C{eqType},
namely \C{Choice.eqType T}: it tells type inference that \C{Choice.sort T}
can be unified with \lstinline/Equality.sort $\mcbimplm{E}$/ by taking
\lstinline/$\mcbimplm{E}={}$Choice.eqType T/.

The bottom structure in the \mcbMC{} algebraic hierarchy introduced by
the \lib{ssralg} library is \C{zmodType} (\C{GRing.Zmodule.type}); it
encapsulates additive groups, and directly extends the \C{choiceType}
structure.

\begin{coq}{name=packed30}{}
Module GRing.

Module Zmodule.

Record mixin_of (V : Type) : Type := Mixin {
  zero : V; opp : V -> V; add : V -> V -> V;
  _ : associative add;
  ...}.

Record class_of T := Class { base: Choice.class_of T; mixin: mixin_of T }.
Structure type := Pack {sort; _ : class_of sort}.
\end{coq}
\index[coq]{\C{zmodType}}

Strictly speaking, the \mcbMC{} algebraic structures don't really
\emph{have} to extend \C{choiceType}, but it is very convenient that
they do.  We can use \C{eqType} and \C{choiceType} operations to test
for 0 in fields, or choose a basis of a subspace, for example.
Furthermore, this is essentially a free assumption, because the
\mcbMC{} algebra mixins specify \emph{strict} identities, such as
\C{associative add} on line 7 above. In the pure \mcbCIC{}, these can only
be realized for concrete data types with a binary representation,
which are both discrete and countable, hence are \C{choiceType}s.  On
the other hand, the ``classical \mcbCIC{}'' axioms needed to construct,
e.g., real numbers, imply that all types are \C{choiceType}s.

Similarly to the definition of \lstinline/eq_op/ in \lib{eqtype}, the
operations afforded by \C{zmodType} are defined just after the
\C{Zmodule} module.

\begin{coq}{name=packed30}{}
Definition zero V := Zmodule.zero (Zmodule.class V).
Definition opp V := Zmodule.opp (Zmodule.class V).
Definition add V := Zmodule.add (Zmodule.class V).
Notation "+%R" := (@add V).
\end{coq}

These are defined inside the \C{GRing} module that encloses most of
\lib{ssralg}, and also given the usual arithmetic syntax (\C{0},
\C{- x}, \C{x + y}) in the \lstinline/%R/
scope. Only the notations are
exported from \C{GRing}, as these definitions are intended to remain
private.

\gotcha{If you see an undocumented \lstinline/GRing.something/,
  then you have broken an abstraction barrier}

The next structure in the hierarchy encapsulates nontrivial rings.
Imposing the non-triviality condition $1 \neq 0$ is a compromise:
it greatly simplifies the theory of polynomials (ensuring for instance that
$X$ has degree 1), at the cost of ruling out possibly trivial matrix rings.

\begin{coq}{name=packed40}{}
Module Ring.

Record mixin_of (R : zmodType) : Type := Mixin {
  one : R;
  mul : R -> R -> R;
  _ : associative mul;
  _ : left_id one mul;
  _ : right_id one mul;
  _ : left_distributive mul +%R;
  _ : right_distributive mul +%R;
  _ : one != 0
}.

Record class_of (R : Type) : Type := Class {
  base : Zmodule.class_of R;
  mixin : mixin_of (Zmodule.Pack base)
}.

Structure type := Pack {sort; _ : class_of sort}.
\end{coq}
\index[coq]{\C{ringType}}

Unlike \C{choiceType} and \C{zmodType}, the definition of the
\C{ringType} mixin depends on the \C{zmodType} structure it extends.
Observe how the class definition instantiates the mixin's
\C{zmodType} parameter with a record created on the fly by packing
the representation type with the base class.

This additional complication does not affect the hierarchy
declarations.  These follow exactly the pattern we saw for
\C{choiceType}, except that we have three definitions, one for each of
the three structures \C{ringType} extends. They all look identical,
thanks to the hidden \C{XXX.base} coercions.

\begin{coq}{name=packed41}{}
Definition eqType := Equality.Pack class.
Definition choiceType := Choice.Pack class.
Definition zmodType := Zmodule.Pack class.
\end{coq}

Two structures extend rings independently: \C{comRingType} provides
multiplication commutativity, and \C{unitRingType} provides computable
inverses for all units (i.e., invertible elements) along with a test
of invertibility. These structures are incomparable, and there are reasonable
instances of each: $2\times 2$ matrices over $\mathbb{Q}$ have
computable inverses but do not commute, while polynomials over
$\mathbb{Z}_p$ commute but do not have easily computable inverses.
The definition of \C{comRingType} and \C{unitRingType} follow exactly
the pattern we have seen,
except there is no need for a \C{ComRing.mixin_of} record.

Since there are also rings such as $\mathbb{Z}_p$ that commute
\emph{and} have computable inverses, and properties such as $(x/y)^n =
x ^n / y^n$ that hold only for such rings, \lib{ssralg} provides a
\C{comUnitRingType} structure for them. Although this structure
simultaneously extends two unrelated structures, it is easy to define
using the packed class pattern: we just reuse the \C{UnitRing}
mixin.

\begin{coq}{name=packed50}{}
Module ComUnitRing.

Record class_of (R : Type) : Type := Class {
  base : ComRing.class_of R;
  mixin : UnitRing.mixin_of (Ring.Pack base)
}.

Structure type := Pack {sort; _ : class_of sort}.
\end{coq}
\index[coq]{\C{comRingType}}

Since we construct explicitly the links between structures with the
packed class pattern, the fact that the hierarchy is no longer a tree
is not an issue.

\begin{coq}{name=packed51}{}
Definition eqType := Equality.Pack class.
Definition choiceType := Choice.Pack class.
Definition zmodType := Zmodule.Pack class.
Definition ringType := Ring.Pack class.
Definition comRingType := ComRing.Pack class.
Definition unitRingType := UnitRing.Pack class.
Definition com_unitRingType := @UnitRing.Pack comRingType class.
\end{coq}
\index[coq]{\C{unitRingType}}
\index[coq]{\C{com_unitRingType}}

In the above code, lines 1--6 relate the new structure to each of the
six structures it
extends, just as before. Line 7 is needed because \C{comUnitRingType}
has several direct ancestors in the hierarchy.  Making
\lstinline/ComUnitRing.com_unitRingType/ a canonical \C{unitRingType}
instance tells type inference that it can unify
\lstinline/UnitRing.sort $\mcbimplm{U}$/ with
\lstinline/ComRing.sort $\mcbimplm{C}$/, by unifying $\mcbimplm{U}$ with
\lstinline/ComUnitRing.com_unitRingType $\mcbimplm{R}$/ and
$\mcbimplm{C}$ with
\lstinline/ComUnitRing.comRingType $\mcbimplm{R}$/,
where \lstinline/$\mcbimplm{R}$ : comUnitRingType/
is a fresh unification variable.
In other words, the \lstinline/ComUnitRing.com_unitRingType/ instance
says that \C{comUnitRingType} is the join of
\C{comRingType} and \C{unitRingType} in the structure hierarchy (see
also~\cite[section 5]{mahboubi:hal-00816703}).

If a new structure $S$ extends structures that are further apart in
the hierarchy more than one such additional link may be needed:
precisely one for each pair of structures whose join is~$S$. For example,
\C{unitRingAlgebraType} requires three such links, while
\C{finFieldType} in library \C{finalg} requires 11. It is highly
advisable to map out the hierarchy when simultaneously extending
multiple structures.

Finally, the telescope and packed class design patterns are not
at all incompatible: it is possible to extend a packed class
hierarchy with telescopes (library \C{fingroup} does this), or to add
explicit ``join'' links to a telescope hierarchy (\lib{ssralg} does this
for its algebraic predicate hierarchy).

%%%%%%%%%%%%%%%%%%%%%%%%%%%%%%%%%%%%%%%%%%%%%%%%%%%%%%%%%%%%%%%%%%%
\mcbLEVEL{3}
\mcbLEARN{Hiding inferred structures in parameters and constructors}
\mcbPROVIDE{\lstinline/phantom/ idioms}
\mcbREQUIRE{\lstinline/GRing/}
\mcbsection{Parameters and constructors}\label{sec:phant}
%%%%%%%%%%%%%%%%%%%%%%%%%%%%%%%%%%%%%%%%%%%%%%%%%%%%%%%%%%%%%%%%%%%

We have noted already that structure instances are often hard to
provide explicitly because it is intended they always be inferred.
For example the explicit \C{ringType} structure for \lstinline/int * rat/ is

\begin{coq}{name=phant1}{}
pair_ringType int_ringType rat_ringType.
\end{coq}

Inference usually happens when an element $x$ of the structure is
passed explicitly; unifying the actual type of $x$ with its expected
type --- the sort of the unspecified structure --- then triggers the
search for a canonical instance.
Unfortunately there are two common situations where a structure is
required and no element is at hand:
\begin{itemize}
\item in a type parameter
\item when constructing an instance explicitly.
\end{itemize}
The first case occurs in \lib{ssralg} for structure types for modules
and algebras, which depend on a ring of scalars: we would like to
specify the type of scalars, and infer its \C{ringType}.  We have seen in
section~\ref{sec:phantom} how to do this, using the \C{phantom} type

\begin{coq}{name=phant4}{}
Inductive phantom T (p : T) := Phantom.
Arguments phantom : clear implicits.
\end{coq}

Here we can use a simpler type, equivalent to \C{phantom Type}

\begin{coq}{name=phant2}{}
Inductive phant (p : Type) := Phant.
\end{coq}

In the definition of the structure \C{type} for left modules, which depends on
\C{ringType} parameter~\C{R}, we add a dummy \C{phant R} parameter \C{phR}.

\begin{coq}{name=phant3}{}
Module Lmodule.

Variable R : ringType.

Record mixin_of (R : ringType) (V : zmodType) := Mixin {
  scale : R -> V -> V;
  ...}.

Record class_of V := Class {
  base : Zmodule.class_of V;
  mixin : mixin_of R (Zmodule.Pack base)
}.

Structure type (phR : phant R) := Pack {sort; _ : class_of sort}.
\end{coq}

Then the \C{Phant} constructor readily yields a value for \C{phR},
from just the sort of \C{R}. Hiding the call to \C{Phant} in
a \lstinline/Notation/

\begin{coq}{name=phant4}{}
Notation lmodType R := (type (Phant R)).
\end{coq}

allows us to write \C{V : lmodType (int * rat)} and let type inference
fill in the unsightly expression
\lstinline/pair_ringType int_ringType rat_ringType/ for \C{R}.

Inference for constructors is more involved, because it has to produce
bespoke classes and mixins subject to dependent typing constraints.
While it is in principle possible to program this using dependent
matching and transport, the complexity of doing so can be daunting.

Instead, we propose a simpler, static solution using a combination
of phantom and function types:

\begin{coq}{name=phant20}{}
Definition phant_id T1 T2 v1 v2 := phantom T1 v1 -> phantom T2 v2.
\end{coq}

For example, each packed class contains exactly the same definition of
the clone constructor, following the introduction of
section variables \C{T} and~\C{cT}, and the definition of \C{class}:

\begin{coq}{name=phant21}{}
Definition clone c & phant_id class c := @Pack T c.
\end{coq}

Recall that with the {\sc SSReflect} extension to \Coq{}, \lstinline/& T/
introduces an anonymous parameter of type~\C{T}. As for \C{lmodType}
above we use \lstinline/Notation/ to supply the identity function
for this dummy functional parameter

\begin{coq}{name=phant22}{}
Notation "[ 'choiceType' 'of' T ]" := (@clone T _ _ id).
\end{coq}

In the context of \lstinline/Definition NN := nat/,
\lstinline/[choiceType of NN]/ will by construction return a \C{choiceType}
instance with sort \emph{exactly} \C{NN} --- provided it is well typed.

Now type checking
\lstinline/(@clone NN _ _ id)/ will try to give
\lstinline/id${}\equiv{}$(fun x => x)/ the type
\lstinline/(phant_id (Choice.class $\mcbimplm{cT}$) $\mcbimplm{c}$)/.
It will assign \C{x} the type
\lstinline/(phantom (Choice.class_of (Choice.sort $\mcbimplm{cT}$)) (Choice.class $\mcbimplm{cT}$))/,
which it will then unify with \lstinline/(phantom (Choice.class_of NN) $\mcbimplm{c}$)/.
To do so \lstinline/Choice.sort $\mcbimplm{cT}$/
will first be unified with \C{NN},
by setting $\mcbimplm{cT}$ to the canonical instance
\lstinline/nat_choiceType/ found by unfolding the definition of~\C{NN},
then setting \lstinline/$\mcbimplm{c}$/
to \lstinline/Choice.class nat_choiceType/.

The code for the instance constructor for \C{choiceType} is almost
identical, because it only extends \C{eqType} with a mixin that does
not depend on \C{eqType}.
Note that this definition allows \Coq{} to infer \C{T} from \C{m}.

\begin{coq}{name=phant23}{}
Definition pack T m :=
  fun bT b & phant_id (Equality.class bT) b => Pack (@Class T b m).
Notation ChoiceType T m := (@pack T m _ _ id).
\end{coq}

For \C{ringType} we use a second \lstinline/phant_id/ \C{id} parameter
to check the dependent type constraint on the mixin.

\begin{coq}{name=phant24}{}
Definition pack T b0 (m0 : mixin_of (@Zmodule.Pack T b0)) :=
  fun bT b & phant_id (Zmodule.class bT) b =>
  fun    m & phant_id m0 m => Pack (@Class T b m).
Notation RingType T m := (@pack T _ m _ _ id _ id).
\end{coq}

Type-checking the second \C{id} will set \C{m} to \C{m0} after
checking that the inferred base class \lstinline/b $\equiv$ Zmodule.class bT/
coincides with the actual base class \C{b0} in the structure parameter
of the type of \C{m0}.  Forcing the sort of that parameter to be equal
to \C{T} allows \Coq{} to infer \C{T} from~\C{m}.

The instance constructor for the join structure
\C{comUnitRingType} uses a similar projection-by-unification idiom to
extract a mixin of the appropriate type from the inferred
\C{unitRingType} of a given type~\C{T}. This is the only constructor
for \C{comUnitRingType}.

\begin{coq}{name=phant25}{}
Definition pack T :=
  fun bT b & phant_id (ComRing.class bT) (b : ComRing.class_of T) =>
  fun mT m & phant_id (UnitRing.class mT) (@UnitRing.Class T b m) =>
  Pack (@Class T b m).
Notation "[ 'comUnitRingType' 'of' T ]" := (@pack T _ _ id _ _ id).
\end{coq}

Finally, the instance constructor for the left algebra structure
\C{lalgType}, a join structure with an additional axiom and a
\C{ringType} parameter, uses all the patterns discussed in this section,
using a \C{phant} and three \lstinline/phant_id/ arguments.

\begin{coq}{name=phant26}{width=12.5cm}
Definition pack T b0 mul0 (axT: @axiom R (@Lmodule.Pack R _ T b0 T) mul0):=
 fun bT b & phant_id (Ring.class bT) (b : Ring.class_of T) =>
 fun mT m & phant_id (@Lmodule.class R phR mT) (@Lmodule.Class R T b m) =>
 fun ax & phant_id axT ax =>
 Pack (Phant R) (@Class T b m ax) T.
...
Notation LalgType R T a := (@pack _ (Phant R) T _ _ a _ _ id _ _ id _ id).
\end{coq}
The interested reader can also refer to~\cite[section 7]{mahboubi:hal-00816703}
for a description of this technique.

%%%%%%%%%%%%%%%%%%%%%%%%%%%%%%%%%%%%%%%%%%%%%%%%%%%%
\section{Linking a custom data type to the library}
\index[concept]{provide eqType structure}
\index[concept]{provide choiceType structure}
\index[concept]{provide finType structure}

The sub-type kit of chapter~\ref{ch:sigmabool} is not the only
way to easily add instances to the library.  For example
imagine we are interested to define the type of a wind rose and attach
to it the theory of finite types.

\begin{coq}{}{}
Inductive windrose := N | S | E | W.
\end{coq}
The most naive way to show that \C{windrose} is a \C{finType} is
to provide a comparison function, then a choice function, \ldots
finally an enumeration.  Instead, it is much simpler to show one
can punt \C{windrose} in bijection with a pre-existing finite type,
like \C{'I_4}.  Let us start by defining the obvious injections.

\begin{coq}{}{}
Definition w2o (w : windrose) : 'I_4 :=
  match w with
  | N => inord 0 | S => inord 1 | E => inord 2 | W => inord 3
  end.
\end{coq}
Remark how the \C{inord} constructor lets us postpone
the (trivial by computation) proofs that 0, 1, 2, 3
are smaller than 4.

The type of ordinals is larger; hence we provide only a partial
function.

\begin{coq}{}{}
Definition o2w (o : 'I_4) : option windrose :=
  match val o with
  | 0 => Some N | 1 => Some S | 2 => Some E | 3 => Some W
  | _ => None
  end.
\end{coq}
Then we can show that these two functions cancel out.

\begin{coq}{}{}
Lemma pcan_wo4 : pcancel w2o o2w.
Proof. by case; rewrite /o2w /= inordK. Qed.
\end{coq}
Now, thanks to the \C{PcanXXXMixin} family of lemmas, one
can inherit on \C{windrose} the structures of ordinals.

\begin{coq}{}{}
Definition windrose_eqMixin := PcanEqMixin pcan_wo4.
Canonical windrose_eqType := EqType windrose windrose_eqMixin.
Definition windrose_choiceMixin := PcanChoiceMixin pcan_wo4.
Canonical windrose_choiceType := ChoiceType windrose windrose_choiceMixin.
Definition windrose_countMixin := PcanCountMixin pcan_wo4.
Canonical windrose_countType := CountType windrose windrose_countMixin.
Definition windrose_finMixin := PcanFinMixin pcan_wo4.
Canonical windrose_finType := FinType windrose windrose_finMixin.
\end{coq}
Only one tiny detail is left on the side.  To use \C{windrose}
in conjunction with \C{\\in} and \C{#|...|}, the type declaration has
to be tagged as a \C{predArgType} as follows.
\index[coq]{\C{preArgType}}

\begin{coq}{}{}
Inductive windrose : predArgType := N | S | E | W.
\end{coq}
After that, our new data type can be used exactly as the ordinal one
can.

\begin{coq}{}{}
Check (N != S) && (N \in windrose) && (#| windrose | == 4).
\end{coq}

More in general a data type can be equipped with a
\C{eqType}, \C{choiceType}, and \C{countType} structure by
providing a correspondence with the generic tree data type
\C{(GenTree.tree T)}: an n-ary tree with nodes labelled with
natural numbers and leafs carrying a value in \C{T}.
\index[coq]{\C{GenTree.tree}}

%%%%%%%%%%%%%%%%%%%%%%%%%%%%%%%%%%%%%%%%%%%%%%%%%%%%%%%%%%%%%%%%%%%
%\mcbLEVEL{2}
%\mcbLEARN{the collection hierarchy; inheritance by coercion; keyed predicates}
%\mcbPROVIDE{\lstinline/.. \in ../}
%\mcbREQUIRE{eqType, seq, coercions}
%\mcbsection{Predicate hierarchies}
%%%%%%%%%%%%%%%%%%%%%%%%%%%%%%%%%%%%%%%%%%%%%%%%%%%%%%%%%%%%%%%%%%%
%
%in\\
%PredType\\
%applicative\_pred and collective\_pred classes\\
%simplPred\\
%mem\\
%bounded quantification
%
%PredArgType\\
%pred\_class\\
%PredArgType\\
%\{set \_\}, \{group \_\}

%closure properties\\

%keyed predicates\\
%default keying
